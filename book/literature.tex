% Добавить ссылки на все наши статьиб препринты и 2 диплома. Если у работы DOI, добавить
% Добавить URL-ы ко всем источникам, по возможности

\begin{thebibliography}{99}
% 1
\bibitem{Pauli} Pauli W., Uber Gasentartung und Paramagnetismus // Zeutschrift für Physik,
1927, v.41, p.81-102. URL: \href{https://link.springer.com/article/10.1007\%2FBF01391920}{https://link.springer.com/article/10.1007\%2FBF01391920}

\bibitem{Sommerfeld} Sommerfeld A., Zur Electronentheore der Metalle aQuf Grundder Fermishen
Statistik // Zeutschrift für Physik, 1928, v.47, p.1-3. URL: \href{https://link.springer.com/article/10.1007\%2FBF01391052}{https://link.springer.com/article/10.1007\%2FBF01391052}

\bibitem{Stoner} Stoner E. C., McDougall J., The computation of fermi-dirac functions //
Philosophical Transactions of the Royal Society of London. Series A, Mathematical and Physical Sciences, 1938, v237(773), p.67-104. URL: \href{https://scinapse.io/papers/2024538905}{https://scinapse.io/papers/2024538905}

\bibitem{Kalitkin1968} Калиткин Н. Н., О вычислении функций Ферми–Дирака, Ж. вычисл. матем. и матем. физ., 1968, том 8, № 1, c.173–175. URL: \href{http://www.mathnet.ru/php/archive.phtml?wshow=paper&jrnid=zvmmf&paperid=7223&option_lang=rus}{$http://www.mathnet.ru/php/archive.phtml?wshow=paper\&jrnid=zvmmf\&paperid=7223\&option_lang=rus$}

\bibitem{Glasser} Glasser M. L., Garoni T. M., Frankel N. E. Complete asymptotic expansions of the fermi-dirac integrals. Journal of Mathematical Physics, 42(4):18601868, 2001.

\bibitem{Kirzhniz} Киржниц Д.А. ЖЭТФ, 1956-58

\bibitem{Kalitkin_Kolganov_ExpConvDan} Калиткин Н.Н., Колганов С.А., КВАДРАТУРНЫЕ ФОРМУЛЫ С ЭКСПОНЕНЦИАЛЬНОЙ СХОДИМОСТЬЮ И ВЫЧИСЛЕНИЕ ФУНКЦИЙ ФЕРМИ–ДИРАКА ДАН 

\bibitem{Kalitkin_Kolganov_ExpConvMatMod} Калиткин Н.Н., Колганов С.А., Вычисление функций Ферми–Дирака экспоненциально сходящимися квадратурами, Матем. моделирование, 29:12(2017),134–146; URL: \href{http://mi.mathnet.ru/rus/mm/v29/i12/p134}{http://mi.mathnet.ru/rus/mm/v29/i12/p134}
Kalitkin N.N, Kolganov S.A., Computing the Fermi-Dirac Functions by Exponentially Convergent Quadratures, Math. Models Comput. Simul.,10:4(2018),472-482;

\bibitem{Kalitkin_Kolganov_Preprint1} Калиткин Н.Н., Колганов С.А. Функции Ферми-Дирака. I. Свойства
функций., 2018, v.41, p.81-102.
URL: \href{https://link.springer.com/article/10.1007\%2FBF01391920}{https://link.springer.com/article/10.1007\%2FBF01391920}
% 10
\bibitem{Kalitkin_Kolganov_PrecApprox} Калиткин Н. Н., Колганов С. А., "Прецизионные аппроксимации функций Ферми–Дирака целого индекса", Матем. моделирование, 28:3(2016), 23–32.
Kalitkin N. N., Kolganov S. А. Precision approximation of the Fermi-Dirak functions of integer index. Mathematical models and computer experiments, 2016, v. №6.

\bibitem{Cody_Thatcher} Thacher Jr H. C. Cody W. J. Rational chebyshev approximations for fermi-dirac integrals of orders -1/2, 1/2 and 3/2. Mathematics of Computation, pages 30-40, 1967.

\bibitem{Theiler} Theiler J. et al. Galassi M., Davies J. GNU scientific library. Network Theory Ltd., 2002

\bibitem{Kalitkin_Kuzmina_Preprint} Калиткин Н.Н., Кузьмина Л.В., Интерполяционные формулы для функций Ферми–Дирака // Препринты ИПМ АН СССР, 1972, № 62.

\bibitem{Kalitkin_Kuzmina_Zhvm} Калиткин Н.Н., Кузьмина Л.В. Интерполяционные формулы для функций Ферми-Дирака. Журнал вычислительной математики и математической физики, 1975, 15, №3, 768-771.

\bibitem{Kuzmina_Disser} Кузьмина Л.В., Численный расчет термодинамических функций веществ в статистической модели атома с квантово-обменными поправками, канд. дисс., Москва, ИПМ АН СССР, 1978.

\bibitem{IEEE_754} IEEE Std 754-1985. IEEE Standard for Binary Floating-Point Arithmetic. \linebreak
URL: \href{https://www.ime.unicamp.br/~biloti/download/ieee_754-1985.pdf}{$https://www.ime.unicamp.br/~biloti/download/ieee_754-1985.pdf$}

\end{thebibliography}




% Бакалаврский диплом

3. Lundstrom M., Kim R. Notes on fermi-dirac integrals. Arxiv preprint
arXiv:0811.0116, 2008.
4. Н.Н. Калиткин. О вычислении функций Ферми-Дирака. Журнал вычисли-
тельной математики и математической физики, 8(1):173-175, 1968.
N.N. Kalitkin. About computation of functions the Fermi-Dirak. Magazine of
computational mathematics and mathematical physics, 8(1):173-175, 1968.
5. Cloutman L. D. Numerical evaluation of the fermi-dirac integrals. The Astrophysical
Journal Supplement Series, 71:677, 1989.
6. Goano M. Algorithm 745: computation of the complete and incomplete fermi-dirac
integral. ACM Transactions on Mathematical Software (TOMS),
21(3):221-232, 1995.
7. MacLeod A. J. Algorithm 779: Fermi-dirac functions of order-1/2, 1/2, 3/2, 5/2.
ACM Transactions on Mathematical Software (TOMS), 24(1):1-12, 1998.
8. Н.Н. Калиткин. Численные методы. Физматлит, Москва, 1978.
29
N.N. Kalitkin. Chislennye metody. Fizmatlit, Moscow, 1978.
9. Н.Н. Калиткин, Е.А. Альшина. Численные методы. Кн.1. Численный анализ. «Академия», М., 2013.
N.N. Kalitkin, E.A. Alshina. Chislennye metody. Kn.1. Chislennyi analis.
«Akademiia», M., 2013.
10. Н.Н. Калиткин, А.А. Белов. Аналог метода Ричардсона для логарифмически сходящегося счета на установление //ДАН. Т. 452. № 3. С.261-265.
N.N. Kalitkin, A.A. Belov. Analogue of the Richardson method for logarithmically
converging time marching // Doklady Nathematics. Vol. 88. No. 2. PP. 596-600.

% Магистерский диплом
4. R. B. Dingle. Asymptotic Expansions: Their Derivation and Interpretation.
London: Academic Press, 1973, 521
5. P. V. Halen and D. L. Pulfrey, "Accurate, short series approximations to
Fermi-Dirac integrals of order -1/2, 1/2, 1, 3/2, 2, 5/2, 3, and 7/2," J.Appl.
Phys. 1985. vol. 57, pp. 5271-5274
6. Cloutman L. D. Numerical evaluation of the fermi-dirac integrals. The
Astrophysical Journal Supplement Series, 71:677, 1989.
7. Goano M. Algorithm 745: computation of the complete and incomplete fermi-dirac integral. ACM Transactions on Mathematical Software (TOMS), 21(3):221-232, 1995.
8. MacLeod A. J. Algorithm 779: Fermi-dirac functions of order -1/2, 1/2, 3/2,
5/2. ACM Transactions on Mathematical Software (TOMS), 24(1):1-12,
1998.
9. Theiler J. et al. Galassi M., Davies J. GNU scientific library. Network Theory
Ltd., 2002
12. Toshio Fukushima. Precise and fast computation of Fermi-Dirac integral of
integer and half integer order by piecewise minimax rational approximation.
Applied Mathematics and Computation, 2015, v. 259, Issue C, pp. 708-729.
13. О. Н. Королева, А. В. Мажукин, В. И. Мажукин, П. В. Бреславский.
Аналитическая аппроксимация интегралов Ферми-Дирака полуцелых и
целых порядков. Математическое моделирование. 2016, т.28, № 11, с.
55-63.
14. И. С. Градштейн, И. Б. Рыжик Таблицы интегралов, сумм, рядов и
произведений. 4-е издание, ФМ, Москва, 1963.
15. Н.Н. Калиткин, Е.А. Альшина. Численные методы. Кн.1. Численный
анализ. «Академия», М., 2013. N.N. Kalitkin, E.A.
Alshina. Chislennye metody. Kn.1. Chislennyi analis. «Akademiia», M., 2013
16. Н.Н. Калиткин, Ритус И.В. Гладкие аппроксимации функций
Ферми-Дирака, Журнал вычислительной математики и математической
физики, т. 26, №3, стр. 461-465, Москва, 1986.
17. Н.Н. Калиткин, И.В Ритус. Гладкие аппроксимации функций
Ферми-Дирака. М., Инст. прикл. мат. АН СССР, 1981, Препринт №72,
с. 9.
N.N. Kalitkin, I. V. Ritus. Smooth approximation for Fermi-Dirak functions.
Moscow, Inst. Appl. Math., 1981, Prepr. №72, с. 9.
18. A. A. Белов. "О коэффициентах квадратурных формул
Эйлера–Маклорена", Матем. моделирование, 25:6 (2013), 72–79.
A. A. Belov. About coefficients of Euler-Mcloren quadrature. Mathematical
models and computer experiments, 25:6 (2013), 72–79.

% Preprint 2
3. Бахвалов Н.С., Жидков Н.П., Кобельков Г.М. Численные методы // 3-е
издание, Москва, БИНОМ. Лаборатория знаний, 2004.
4. Калиткин Н.Н. Численные методы // 2-е издание, Санкт-Петербург,
«БХВ-Петербург», 2011.
5. Калиткин Н.Н., Альшина Е.А. Численные методы, книга 1, Численный
анализ // Москва, «Академия», 2013.
6. Калиткин Н.Н., “Квадратуры Эйлера–Маклорена высоких порядков”,
Матем. моделирование, 16:10 (2004), 64–66, URL:
%http://mi.mathnet.ru/rus/mm/v16/i10/p64
7. Белов А. А., О коэффициентах квадратурных формул Эйлера-
Маклорена, Математическое моделирование, 25:6(2013), 72-79. URL:
%http://mi.mathnet.ru/rus/mm/v25/i6/p72
Belov A. A., Coefficients of Eiler-Maclaurin formulas for numerical
integration. Mathematical models and computer experiments, Jan 2014, Vol 6, Issue
1, pp 32-37.
8. Градштейн И. С., Рыжик И. Б. Таблицы интегралов, сумм, рядов и
произведений. 4-е издание, ФМ, Москва, 1963.
9. Калиткин Н.Н., Колганов С.А., “Вычисление функций Ферми–Дирака
экспоненциально сходящимися квадратурами”, Матем. моделирование, 29:12
(2017),134–146; URL: http://mi.mathnet.ru/rus/mm/v29/i12/p134
%Kalitkin N.N, Kolganov S.A., Computing the Fermi−Dirac Functions by
Exponentially Convergent Quadratures, Math. Models Comput. Simul.,10:4
(2018),472–482;

% Статья 1 
3. M. Lundstrom, R. Kim. Notes on Fermi-Dirac integrals. Arxiv preprint arXiv:0811.0116, 2008.
4. Н.Н. Калиткин. О вычислении функций Ферми-Дирака // Журнал вычислительной математи-
ки и математической физики, 1968, 8(1), c.173-175.
N.N. Kalitkin. About computation of functions the Fermi-Dirak // Magazine of computational mathematics
and mathematical physics, 1968, 8(1), p.173-175.
5. L.D. Cloutman. Numerical evaluation of the Fermi-Dirac integrals // The Astrophysical Journal
Supplement Series, 1989, 71, 677p.
6. M. Goano. Algorithm 745: computation of the complete and incomplete Fermi-Dirac integral //
ACM Transactions on Mathematical Software (TOMS), 1995, 21(3), p.221-232.
7. A.J. MacLeod. Algorithm 779: Fermi-Dirac functions of order -1/2, 1/2, 3/2, 5/2 // ACM Transactions
on Mathematical Software (TOMS), 1998, 24(1), p.1-12.
8. Н.Н. Калиткин. Численные методы. М.: Физматлит, 1978.
N.N. Kalitkin. Chislennye metody. M.: 1978.
9. Н.Н. Калиткин, Е.А. Альшина. Численные методы. Кн.1. Численный анализ. М.: Академия, 2013.
N.N. Kalitkin, E.A. Alshina. Chislennye metody. Kn.1. Chislennyi analis. M.: Akademiia, 2013.
10. Н.Н. Калиткин, А.А. Белов. Аналог метода Ричардсона для логарифмически сходящегося
счета на установление // ДАН, 2013, т.452, № 3, с.261-265.
N.N. Kalitkin, A.A. Belov. Analogue of the Richardson method for logarithmically converging time
marching // Doklady Mathematics, 2013, v.88, №2, p.596-600.

% Статья 2
3. M. Lundstrom, R. Kim. Notes on Fermi-Dirac integrals. Arxiv preprint arXiv:0811.0116, 2008.
4. Н.Н. Калиткин. О вычислении функций Ферми-Дирака // Журнал вычислительной математики и математической физики, 1968, 8(1):173-175;
N.N. Kalitkin. About computation of functions the Fermi-Dirac // Magazine of computational
mathematics and mathematical physics, 1968, 8(1):173-175.
5. L.D. Cloutman. Numerical evaluation of the Fermi-Dirac integrals // The Astrophysical Journal
Supplement Series, 71:677, 1989.
6. M. Goano. Algorithm 745: computation of the complete and incomplete Fermi-Dirac integral.
ACM Transactions on Mathematical Software (TOMS), 1995, 21(3):221-232.
7. A.J. MacLeod. Algorithm 779: Fermi-Dirac functions of order -1/2, 1/2, 3/2, 5/2. ACM Transactions
on Mathematical Software (TOMS), 1998, 24(1):1-12.
8. Н.Н. Калиткин, С.А. Колганов. Прецизионные аппроксимации функций Ферми–Дирака целого индекса // Матем. моделирование, 2016, т.28, №3, с.23–32;
N.N. Kalitkin, S.A. Kolganov. Pretsizionnye approksimatsii funktsii Fermi–Diraka tselogo indeksa
// Matem. modelirovanie, 2016, t.28, №3, s.23–32
9. Н.Н. Калиткин, И.В. Ритус. Гладкие аппроксимации функций Ферми–Дирака. М.: ИПМ
им. М. В. Келдыша, 1981, препринт №72, 9с.;
N.N. Kalitkin, I.V. Ritus. Smooth approximations of functions the Fermi-Dirac // Magazine of computational
mathematics and mathematical physics. М.: IPM of М.V. Keldysh, 1981, preprint
№72, 9с.
10. Н.Н. Калиткин, И.В. Ритус. Гладкая аппроксимация функций Ферми–Дирака // Журнал вы-
числительной математики и математической физики, 1986, т.26, №3, с.461-464.
N.N. Kalitkin, I.V. Ritus. Smooth approximation of Fermi–Dirac functions // USSR Computational
Mathematics and Mathematical Physics, 1986, 26:2, 87–89.

% Статья 3 ДАН
1. Бахвалов Н. С., Жидков Н. П., Кобельков Г. М. Численные методы. 3-е изд. М.: БИНОМ / Лаб. знаний, 2004.
2. Калиткин Н. Н. Численные методы. 2-е изд. СПб.: БХВ-Петербург, 2011.
3. Калиткин Н. Н., Альшина Е. А. Численные методы. Кн. 1. Численный анализ. М.: Академия, 2013.
4. Белов А. А. О коэффициентах квадратурных формул Эйлера–Маклорена // Мат. моделирование. 2013. Т. 25. №6. C. 72–79.
6. Калиткин Н. Н., Колганов С. А. Прецизионныеаппроксимации функций Ферми–Дирака це-
лого индекса // Мат. моделирование. 2016. Т. 28.№ 3. С. 23–32.

% Статья 4 

2. И. С. Градштейн, И. Ь. Рыжик Таблицы интегралов, сумм, рядов и произведений. 4-е издание, ФМ,
Москва, 1963.
4. Н. Н. Калиткин, С. А. Колганов, "Уточнение прецизионных аппроксимаций функций Ферми–Дирака
целого индекса", Матем. моделирование, 28:3 (2016), 23–32.
N. N. Kalitkin, S. А. Kolganov. Correction of the precision approximations of the Fermi-Dirak functions of
integer index. Mathematical models and computer experiments, 2016, v. №6.
5. А. А. Белов, "О коэффициентах квадратурных формул Эйлера–Маклорена", Матем. моделирование, 25:6
(2013), 72–79.
A. A. Belov. About coefficients of Euler-Mcloren quadrature. Mathematical models and computer experiments,
25:6 (2013), 72–79.
6. Н.Н. Калиткин, Е.А. Альшина. Численные методы. Кн.1. Численный анализ. «Академия», М., 2013.
N.N. Kalitkin, E.A. Alshina. Chislennye metody. Kn.1. Chislennyi analis. «Akademiia», M., 2013
7. Н.Н. Калиткин, Ритус И.В. Гладкие аппроксимации функций Ферми-Дирака, Журнал вычислительной
математики и математической физики, 3, стр. 461-465, Москва, 1986.
8. Н.Н. Калиткин, Ритус И.В. Гладкие аппроксимации функций Ферми-Дирака. Препринт, Москва, 1986.