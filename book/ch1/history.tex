\S \textbf{1.2. История}
\\

Свойствам функций ФД посвящено лишь несколько работ. Дадим их обзор.

Функции ФД впервые появились на заре развития квантовой механики в работах Паули [1] и Зоммерфельда [2] при описании частично вырожденного электронного газа в металлах. Основные свойства этих функций изложены в статье МакДугала и Стоунера [3]. В ней приведены сходящийся ряд при $x < 0$ и
асимптотическое разложение при $x \to +\infty$ . Заметим, что в асимптотическом разложении в формуле (5.3) имеется опечатка. Вместо сомножителя $(k - r + 2)$ следует писать $(k - 2r + 2)$ .

В работе Калиткина [4] построен ряд, сходящийся при любых значениях x. Однако при $x > 3$ скорость сходимости невелика и фактически этот ряд удобен лишь при $-\infty < x < 1$.

Улучшенное выражение для асимптотического разложения получено в работе Glasser [5]; в нем содержится связь между значениями  $I_k(x)$ и $I_k(-x)$ .

Интегральную функцию ФД ввел Киржниц [6]. Ее разложение в сходящийся ряд при $x < 0$  и асимптотический ряд при $x \to +\infty$ было построено в работах Каликина, Кузьминой, Луцкого и Колганова.

Ряды, построенные в указанных выше работах, позволяют вычислять функции ФД с высокой точностью при $x ~< 0$ и при $x >~ 50$. Остается решить вопрос о практическом вычислении функций ФД в диапазоне $0 ~<x <~ 50$.

Функции ФД в указанном диапазоне(и вообще при любом аргументе $x$) и при заданном $k$ можно найти с
точностью ошибок компьютерного округления, непосредственно вычисляя
интеграл (1) по каким-либо квадратурным формулам на достаточно
подробной сетке. Правда такое вычисление по обычным классическим квадратурным формулам чрезмерно трудоемко и его нецелесообразно использовать как компьютерную подпрограмму. С его помощью можно составить подробные многозначные таблицы этих функций. Первый пример такой таблицы был приведен уже в [3]. Однако такие таблицы имеют огромный объем и также
непригодны для создания компьютерных
подпрограмм.

Правда, в последние годы для функций ФД полуцелого индекса Калиткиным и Колгановым разработаны экспоненциально сходящиеся квадратуры, радикально уменьшающие трудоемкость вычислений [7-9]. Они уже пригодны для создания попрограмм, хотя их трудоемкость ощутимо больше, чем для вычислений с помощью упомянутых выше рядов.

Поэтому остается актуальной задача построения быстрых компьютерных подпрограмм. Вот почему
ряд работ был посвящен вычислению таблиц функций для конкретных
индексов и построению для них несложных экономичных аппроксимаций. В них выбирают некоторый
разумный вид аппроксимирующих формул с достаточным числом свободных
параметров. Эти параметры подбирают для получения наилучшей
точности аппроксимации в некоторой норме. Наиболее разумным
представляется получение наилучшей относительной точности в норме C.

Такие аппроксимации для функций целых и полуцелых индексов
строились в цитированных работах. Для функций целых индексов
аппроксимации с 16 верными десятичными знаками построены в [10], что в
полной мере исчерпывает проблему при расчетах с 64-битовыми
числами. Для функций полуцелых индексов опубликованные
аппроксимации имеют меньшую точность. 

В работе МакДугала и Стоунера [3] рассмотрены способы вычисления функций ФД индексов $k =3/2, 1/2, -1/2$. Но эти способы ещё слишком сложны и трудоемки. Они непригодны для практического использования.

% TODO: 50 и 55 года

В работе Коди и Тетчера [11] для функций ФД индексов $k =3/2, 1/2, -1/2$ построен набор аппроксимаций, составленных из трех кусков. Вид этих формул выбран удачно. При $x \leqslant +1$ аппроксимация является отношением двух многочленов от $e^x$; она точно передает главный член асимптотики при $x \to -\infty$. При $x \geqslant +4$ аппроксимация есть произведение множителя $x^{k+1}$ на отношение многочленов от $x^{-2}$. Она точно передает главный член асимптотики при $x \to +\infty$. В промежутке $+1 \leqslant x \leqslant +4$ используется отношение двух многочленов одинаковой степени от $x$. Наборы с небольшим числом коэффициентов обеспечивают невысокую точность. Для набора с наибольшим числом коэффициентов заявлена относительная точность $10^{-12}$. К сожалению, фактическая точность существенно хуже. В Табл.1 приведены относительные рассогласованности смежных формул на стыках. Они на много порядков превышают заявленную погрешность, особенно для $k = 3/2$. По-видимому, в табличных коэффициентах имеются опечатки. Поэтому на практике использовать эти формулы невозможно. 

В работе Theiler [12] построены прецизионные аппроксимации для функций ФД индексов $k = 3/2, 1/2 -1/2$. В ней при $x \leqslant -1$ используется классический ряд по степеням $e^x$. При $x \geqslant 30$ использован асимптотический ряд. Промежуток разбивается на 4 отрезка и на каждом отрезке табулированная функция аппроксимируется полиномами Чебышева. Окончательная относительная погрешность оценивается в $~10^{-14}$. Сама по себе такая точность хороша. Однако вид аппроксимации выбран менее удачно, чем в работе Коди и Тэтчера. Общее число коэффициентов аппроксимации из-за этого велико. Использование чебышевских многочленов не обеспечивает точность производных. Кроме того, на практике требуются функции и других полуцелых индексов.

Построить единую аппроксимирующую формулу высокой точности
вряд ли представляется возможным. 

Другой принцип построения формул был предложен в работах Калиткина и Кузьминой [13-15]. В них различные функции ФД выражались через функцию $I_{-1/2}(x)$. Это позволило построить аппроксимацию из 2 кусков. При построении левой части использовались качественные соображения поведения при $x \to \infty$, а для правой части - асимптотики при $x \to +\infty$. Это позволило при небольшом числе коэффициентов получить относительную погрешность до $3·10^{-8}$ для функций полуцелых индексов и $7·10^{-6}$ для интегральной функции ФД.

% TODO: современные рещультаты.