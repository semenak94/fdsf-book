\chapter{ВВЕДЕНИЕ}

% Параграф \S или \textsection
\S \textbf{1.1. Определение функций}
\\

Функции Ферми-Дирака ( далее ФД ) возникают в задачах квантовой механики при описании свойств вещества, обусловленных поведением электронов или других фермионов. Как известно, функция распределения электронов в 6-мерном пространстве импульсов $\textbf{p}$ и координат $\textbf{r}$ имеет следующий вид:
\begin{equation}
f(\textbf{p},\textbf{r})=\frac{const}{1+\exp{(\frac{\frac{1}{2}\textbf{p}^2+\phi(\textbf{r})-\mu}{T})}}.
\label{eq:ref_1_1_1}
\end{equation}
Здесь $\phi(\textbf{r})$ есть потенциал электрического поля, $\mu$ – химический потенциал, а $T$ – температура. Константа есть статистический вес частицы, определяемый её спином; для электрона $const = 2$. Все физические формулы написаны в атомной системе единиц; за единицу берутся масса электрона, заряд электрона и постоянная Планка.
\indent
\newline
При решении квантово-механических задач используются различные моменты фермиевского распределения. Они равны сверткам различных степеней импульса $\textbf{p}$ с этим распределением по объёму в импульсном пространстве $d\textbf{p}=4\pi^2dp$. Например, электронная плотность в пространстве координат есть свертка нулевой степени импульса:
\begin{equation}
\rho (\textbf{r}) = \int f(\textbf{p}) d \textbf{p} = \int\limits_0^{\infty} \frac{8\pi^2 dp}{1 +\exp[\frac{\frac{1}{2} \textbf{p}^2 + \varphi(\textbf{r}) - \mu}{T}]}.
\label{eq:ref_1_1_2}
\end{equation}
Плотность кинетической энергии есть свертка квадрата импульса:
\begin{equation}
E(\textbf{r}) = \int(\frac{1}{2}p^2*f(\textbf{p}d\textbf{p}) = \int\limits_0^{\infty} \frac{4\pi p^4dp}{1+\exp[(\frac{1}{2}\textbf{p}^2 + \varphi(\textbf{r}) - \mu) / T]}.
\label{eq:ref_1_1_3}
\end{equation}
Нахождение проводимости требует нахождения потока частиц, т.е. умножения $f(\textbf{p}, \textbf{r})$ на импульс $\textbf{p}$; это эквивалентно введению в числитель подынтегрального выражения ($\ref{eq:ref_1_1_2}$) степени $p^3$ вместо $p^2$. Теплопроводность определяется через поток кинетической энергии; это эквивалентно введению в числитель подынтегрального выражения ($\ref{eq:ref_1_1_3}$) степени $p^5$ вместо $p^4$.

В таких свертках принято делать следующую замену переменных:
\begin{equation}
t = \frac{p^2}{2T}, x = \frac{\mu - \phi(\textbf{r})}{T}.
\label{eq:ref_1_1_4}
\end{equation}
Тогда различные моменты с точностью до константы приобретают следующий вид:
\begin{equation}
I_k(x)=\int\limits_0^{\infty} \frac{t^kdt}{1+exp(t-x)}, x \in (-\infty ;+\infty).
\label{eq:ref_1_1_5}
\end{equation}
Например, $k = 1/2$ для электронной плотности ($\ref{eq:ref_1_1_2}$), $k = 1$ для потока частиц, $k = 3/2$ для плотности кинетической энергии ($\ref{eq:ref_1_1_3}$) и $k = 2$ для плотности потока энергии(электронной теплопроводности). Целые $k$ сooтветствуют нечетным моментам импульса, а полуцелые – четным.

В математической теории функций ФД рассматриваются произвольные индексы $k$ и комплексные значения $x$.

Интеграл в ($\ref{eq:ref_1_1_5}$) сходится при $k > -1$. Способ доопределения функции ФД для индекса $k < -1$ будет показан далее. Заметим, что при целых отрицательных $k$ функции ФД имеют полюс.

В физических задачах нужны только целые и полуцелые индексы и вещественные значения $x$. Вычислению функций таких индексов в основном посвящена данная книга. Однако многие приведенные далее выражения справедливы для произвольных индексов $k$.

В квантово-механических моделях атома возникает ещё одна специфическая функция, связанная
с вычислением обменной энергии в квазиклассическом приближении. Пояним подробнее.
При квантово-механических расчетах атома широко используется приближение Хартри–Фока, при котором в 
многоэлектронном уравнении Шредингера многоэлектронная волновая функция заменяется не произведением 
электронных функций (как в приближении Хартри), а детерминантом,
составленным из одноэлектронных функций. Это эффективно учитывает
обменное взаимодействие. При этом к электростатическому взаимодействию
электронов добавляется дополнительное слагаемое, называемое потенциалом
обменного взаимодействия. Однако, такое приближение ещё очень сложно для
численных расчетов.

Вычисления обменного потенциала в квазиклассическом приближении
дает гораздо более простое выражение. Такое приближение называют
приближением Слэтера, который первым написал его для нулевой
температуры. При ненулевой температуре возникает следующая функция:
\begin{equation}
J(x)=\int\limits_{-\infty}^x [I_{-\frac{1}{2}}(\xi)]^2d\xi.
\label{eq:ref_1_1_6}
\end{equation}
Её называют интегральной функцией Ферми-Дирака. Её свойства и способы вычисления также будут рассмотрены в этой книге. 
\\

\S \textbf{1.2. История}
\\

Свойствам функций ФД посвящено лишь несколько работ. Дадим их обзор.

Функции ФД впервые появились на заре развития квантовой механики в работах Паули \cite{Pauli} и Зоммерфельда \cite{Sommerfeld} при описании частично вырожденного электронного газа в металлах. Основные свойства этих функций изложены в статье МакДугала и Стоунера \href{Stoner}. В ней приведены сходящийся ряд при $x < 0$ и асимптотическое разложение при $x \to +\infty$ . Заметим, что в асимптотическом разложении в формуле (5.3) имеется опечатка. Вместо сомножителя $(k - r + 2)$ следует писать $(k - 2r + 2)$ .

В работе Калиткина \cite{Kalitkin1968} построен ряд, сходящийся при любых значениях x. Однако при $x > 3$ скорость сходимости невелика и фактически этот ряд удобен лишь при $-\infty < x < 1$.

Улучшенное выражение для асимптотического разложения получено в работе \cite{Glasser}; в нем содержится связь между значениями  $I_k(x)$ и $I_k(-x)$ .

Интегральную функцию ФД ввел Киржниц \cite{Kirzhniz}. Ее разложение в сходящийся ряд при $x < 0$  и асимптотический ряд при $x \to +\infty$ было построено в работах Калиткина, Кузьминой, Луцкого и Колганова.

Ряды, построенные в указанных выше работах, позволяют вычислять функции ФД с высокой точностью при $x ~< 0$ и при $x >~ 50$. Остается решить вопрос о практическом вычислении функций ФД в диапазоне $0 ~<x <~ 50$.

Функции ФД в указанном диапазоне(и вообще при любом аргументе $x$) и при заданном $k$ можно найти с
точностью ошибок компьютерного округления, непосредственно вычисляя
интеграл (1) по каким-либо квадратурным формулам на достаточно
подробной сетке. Правда такое вычисление по обычным классическим квадратурным формулам чрезмерно трудоемко и его нецелесообразно использовать как компьютерную подпрограмму. С его помощью можно составить подробные многозначные таблицы этих функций. Первый пример такой таблицы был приведен уже в \cite{Stoner}. Однако такие таблицы имеют огромный объем и также непригодны для создания компьютерных подпрограмм.

Правда, в последние годы для функций ФД полуцелого индекса Калиткиным и Колгановым разработаны экспоненциально сходящиеся квадратуры, радикально уменьшающие трудоемкость вычислений \cite{Kalitkin_Kolganov_ExpConvDan,Kalitkin_Kolganov_ExpConvMatMod,Kalitkin_Kolganov_Preprint1}. Они уже пригодны для создания попрограмм, хотя их трудоемкость ощутимо больше, чем для вычислений с помощью упомянутых выше рядов.

Поэтому остается актуальной задача построения быстрых компьютерных подпрограмм. Вот почему
ряд работ был посвящен вычислению таблиц функций для конкретных
индексов и построению для них несложных экономичных аппроксимаций. В них выбирают некоторый
разумный вид аппроксимирующих формул с достаточным числом свободных
параметров. Эти параметры подбирают для получения наилучшей
точности аппроксимации в некоторой норме. Наиболее разумным
представляется получение наилучшей относительной точности в норме $C$.

Такие аппроксимации для функций целых и полуцелых индексов
строились в цитированных работах. Для функций целых индексов
аппроксимации с 16 верными десятичными знаками построены в \cite{KalitkinKolganov_PrecApprox}, что в
полной мере исчерпывает проблему при расчетах с 64-битовыми
числами. Для функций полуцелых индексов опубликованные
аппроксимации имеют меньшую точность. 

В работе МакДугала и Стоунера \cite{Stoner} рассмотрены способы вычисления функций ФД индексов \linebreak $k =3/2, 1/2, -1/2$. Но эти способы ещё слишком сложны и трудоемки. Они непригодны для практического использования.

% TODO: 50 и 55 года

В работе Коди и Тетчера \cite{Cody_Thatcher} для функций ФД индексов $k =3/2, 1/2, -1/2$ построен набор аппроксимаций, составленных из трех кусков. Вид этих формул выбран удачно. При $x \leqslant +1$ аппроксимация является отношением двух многочленов от $e^x$; она точно передает главный член асимптотики при $x \to -\infty$. При $x \geqslant +4$ аппроксимация есть произведение множителя $x^{k+1}$ на отношение многочленов от $x^{-2}$. Она точно передает главный член асимптотики при $x \to +\infty$. В промежутке $+1 \leqslant x \leqslant +4$ используется отношение двух многочленов одинаковой степени от $x$. Наборы с небольшим числом коэффициентов обеспечивают невысокую точность. Для набора с наибольшим числом коэффициентов заявлена относительная точность $10^{-12}$. К сожалению, фактическая точность существенно хуже. В Табл.1 приведены относительные рассогласованности смежных формул на стыках. Они на много порядков превышают заявленную погрешность, особенно для $k = 3/2$. По-видимому, в табличных коэффициентах имеются опечатки. Поэтому на практике использовать эти формулы невозможно. 

В работе Theiler \cite{Theiler} построены прецизионные аппроксимации для функций ФД индексов $k = 3/2, 1/2 -1/2$. В ней при $x \leqslant -1$ используется классический ряд по степеням $e^x$. При $x \geqslant 30$ использован асимптотический ряд. Промежуток разбивается на 4 отрезка и на каждом отрезке табулированная функция аппроксимируется полиномами Чебышева. Окончательная относительная погрешность оценивается в $~10^{-14}$. Сама по себе такая точность хороша. Однако вид аппроксимации выбран менее удачно, чем в работе Коди и Тэтчера. Общее число коэффициентов аппроксимации из-за этого велико. Использование чебышевских многочленов не обеспечивает точность производных. Кроме того, на практике требуются функции и других полуцелых индексов.

Построить единую аппроксимирующую формулу высокой точности вряд ли представляется возможным. 

Другой принцип построения формул был предложен в работах Калиткина и Кузьминой \cite{Kalitkin_Kuzmina_Preprint,Kalitkin_Kuzmina_Zhvm,Kuzmina_Disser}. В них различные функции ФД выражались через функцию $I_{-1/2}(x)$. Это позволило построить аппроксимацию из 2 кусков. При построении левой части использовались качественные соображения поведения при $x \to \infty$, а для правой части - асимптотики при $x \to +\infty$. Это позволило при небольшом числе коэффициентов получить относительную погрешность до $3 \cdot 10^{-8}$ для функций полуцелых индексов и $7 \cdot 10^{-6}$ для интегральной функции ФД.

% TODO: современные рещультаты.
\S \textbf{1.3. Погрешность округления}

В данной работе рассматриваются методы прямого вычисления функций
ФД целых и полуцелых индексов, а также интегральной функции ФД с
заданной точностью. Желательно вычислять функции ФД с предельно высокой
точностью $\epsilon$, допускаемой компьютером. Поэтому опишем точность, которую
дают распространенные сейчас процессоры при вычислении с плавающей
точкой.

Все операции с плавающей точкой
выполняются арифметическим сопроцессором ( Floating Point Unit ). Его
архитектура фактически не развивается с момента принятия в 1985 году
стандарта IEEE-754 [ссылка], поэтому предельная разрядность чисел остается на
уровне 80 бит. Однако в наиболее распространенных математических
обеспечениях для записи в память используются либо 32-битовые числа (single
precision), либо 64-битовые числа (double precision). В обоих случаях не
полностью используются возможности линейки процессора. В начале 2000-х
годов для языка С++ была возможность использовать 80-битовые числа (long
double precision); в настоящее время эта возможность не поддерживается и мало
где сохранилась. Для суперкомпьютерных вычислений используют 128-
битовые числа; но вычисления делаются программными средствами на тех же
процессорах. Возможно и вычисление с произвольной разрядностью, но они
также выполняются программными средствами. Например, для языка С++ это
библиотека boost::multiprecision.

При вычислениях с плавающей точкой побитовая запись числа выглядит
следующим образом: знак числа – 1 бит, двоичный порядок числа – p бит,
мантисса – m бит. Порядок числа может быть положительным или
отрицательным; но при записи в память к нему автоматически добавляется
положительная константа, равная по модулю максимально возможному
отрицательному порядку, так что в память записывается неотрицательное
число. Это эквивалентно тому, что из p разрядов порядка тратится 1 разряд на
знак порядка и $p - 1$ на модуль порядка.

При записи мантиссы также имеется небольшая "хитрость". Первый
разряд мантиссы всегда равен 1, поэтому при записи он отбрасывается. При
считывании числа в процессор этот разряд автоматически добавляется. Такой
прием позволяет фактически удлинять записываемую мантиссу на 1 бит.

Нетрудно посчитать, что максимально возможный порядок в двоичной
системе есть $2^{p-1} - 1$; для перевода в десятичную систему его нужно умножить
на $lg2$. Разрядность процессора длиннее, чем считанное из памяти число, за
исключением long double. После вычисления результат, записанный на
процессоре, оказывается длиннее отведенного в памяти места. Поэтому при
записи результата в память производится округление. Ошибка такого
округления не превышает половины последнего отброшенного разряда. С
учетом "спрятанного" первого разряда мантиссы это означает, что
относительная ошибка округления есть $\epsilon = 2^{-m-2}$.

В случае long double длина числа равна линейке процессора и нельзя ни
спрятать первый разряд мантиссы, ни иметь лишние разряды для округления; в
этом случае $\epsilon = 2^{-m}$. В Таблице 1 приведены предельные порядки $P = 2^{p-1}lg2$ и
относительные ошибки единичного округления в форме $lg\epsilon$ для
рассмотренных выше случаев вычисления. Видно, что для случая long double
представление числа выбрано не вполне удачно; лучше было бы взять $p = 14,m = 65$.

\begin{table}
\caption{Таблица 1. Компьютерные числа с плавающей точкой}
\begin{center}
\begin{tabular}{|c|c|c|c|c|c|}
\hline
точность & бит & $p$ & $m$ & $P$ & $\lg \epsilon$ \\
\hline
float & 32 & 8 & 23 & $\pm 38$ & -7.5 \\
double & 64 &  11 & 52 & $\pm 308$ & -16.2 \\
long double & 80 & 15 & 64 & $\pm 4932$ & -19.3\\
Quadruple double & 128 & 15 & 112 & $\pm 4932$ & -34.3 \\
\hline
\end{tabular}
\end{center}
\end{table}

Наиболее частыми являются 64-битовые вычисления, поэтому мы будем
ориентироваться на точность $16^10$. Вычисление long double кажутся
заманчивыми, поскольку они повышают точность по сравнению с double без
увеличения машинного времени. Однако разбиение числа на мантиссу и
порядок было недостаточно продумано (лучше было бы перекинуть 1 бит на
мантиссу), так что увеличение точности не столь значительно. Но 32-битовыми
числами не стоит пренебрегать: именно такую точность дают видеокарты,
позволяющие сильно ускорять вычисления за счет конвейерной реализации.