\chapter{ВВЕДЕНИЕ}

% Параграф \S или \textsection
\S \textbf{1.1. Определение функций}
\\

Функции Ферми-Дирака ( далее ФД ) возникают в задачах квантовой механики при описании свойств вещества, обусловленных поведением электронов или других фермионов. Как известно, функция распределения электронов в 6-мерном пространстве импульсов $\textbf{p}$ и координат $\textbf{r}$ имеет следующий вид:
\begin{equation}
f(\textbf{p},\textbf{r})=\frac{const}{1+\exp{(\frac{\frac{1}{2}\textbf{p}^2+\phi(\textbf{r})-\mu}{T})}}.
\label{eq:ref_1_1_1}
\end{equation}
Здесь $\phi(\textbf{r})$ есть потенциал электрического поля, $\mu$ – химический потенциал, а $T$ – температура. Константа есть статистический вес частицы, определяемый её спином; для электрона $const = 2$. Все физические формулы написаны в атомной системе единиц; за единицу берутся масса электрона, заряд электрона и постоянная Планка.
\indent
\newline
При решении квантово-механических задач используются различные моменты фермиевского распределения. Они равны сверткам различных степеней импульса $\textbf{p}$ с этим распределением по объёму в импульсном пространстве $d\textbf{p}=4\pi^2dp$. Например, электронная плотность в пространстве координат есть свертка нулевой степени импульса:
\begin{equation}
\rho (\textbf{r}) = \int f(\textbf{p}) d \textbf{p} = \int\limits_0^{\infty} \frac{8\pi^2 dp}{1 +\exp[\frac{\frac{1}{2} \textbf{p}^2 + \varphi(\textbf{r}) - \mu}{T}]}.
\label{eq:ref_1_1_2}
\end{equation}
Плотность кинетической энергии есть свертка квадрата импульса:
\begin{equation}
E(\textbf{r}) = \int(\frac{1}{2}p^2*f(\textbf{p}d\textbf{p}) = \int\limits_0^{\infty} \frac{4\pi p^4dp}{1+\exp[(\frac{1}{2}\textbf{p}^2 + \varphi(\textbf{r}) - \mu) / T]}.
\label{eq:ref_1_1_3}
\end{equation}
Нахождение проводимости требует нахождения потока частиц, т.е. умножения $f(\textbf{p}, \textbf{r})$ на импульс $\textbf{p}$; это эквивалентно введению в числитель подынтегрального выражения ($\ref{eq:ref_1_1_2}$) степени $p^3$ вместо $p^2$. Теплопроводность определяется через поток кинетической энергии; это эквивалентно введению в числитель подынтегрального выражения ($\ref{eq:ref_1_1_3}$) степени $p^5$ вместо $p^4$.

В таких свертках принято делать следующую замену переменных:
\begin{equation}
t = \frac{p^2}{2T}, x = \frac{\mu - \phi(\textbf{r})}{T}.
\label{eq:ref_1_1_4}
\end{equation}
Тогда различные моменты с точностью до константы приобретают следующий вид:
\begin{equation}
I_k(x)=\int\limits_0^{\infty} \frac{t^kdt}{1+exp(t-x)}, x \in (-\infty ;+\infty).
\label{eq:ref_1_1_5}
\end{equation}
Например, $k = 1/2$ для электронной плотности ($\ref{eq:ref_1_1_2}$), $k = 1$ для потока частиц, $k = 3/2$ для плотности кинетической энергии ($\ref{eq:ref_1_1_3}$) и $k = 2$ для плотности потока энергии(электронной теплопроводности). Целые $k$ сooтветствуют нечетным моментам импульса, а полуцелые – четным.

В математической теории функций ФД рассматриваются произвольные индексы $k$ и комплексные значения $x$.

Интеграл в ($\ref{eq:ref_1_1_5}$) сходится при $k > -1$. Способ доопределения функции ФД для индекса $k < -1$ будет показан далее. Заметим, что при целых отрицательных $k$ функции ФД имеют полюс.

В физических задачах нужны только целые и полуцелые индексы и вещественные значения $x$. Вычислению функций таких индексов в основном посвящена данная книга. Однако многие приведенные далее выражения справедливы для произвольных индексов $k$.

В квантово-механических моделях атома возникает ещё одна
специфическая функция, связанная с вычислением обменной энергии в
квазиклассическом приближении. Пояним подробнее. При квантово-
механических расчетах атома широко используется приближение Хартри–
Фока, при котором в многоэлектронном уравнении Шредингера многоэлектронная волновая функция заменяется не произведением электронных функций (как в приближении Хартри), а детерминантом,
составленным из одноэлектронных функций. Это эффективно учитывает
обменное взаимодействие. При этом к электростатическому взаимодействию
электронов добавляется дополнительное слагаемое, называемое потенциалом
обменного взаимодействия. Однако, такое приближение ещё очень сложно для
численных расчетов.

Вычисления обменного потенциала в квазиклассическом приближении
дает гораздо более простое выражение. Такое приближение называют
приближением Слэтера, который первым написал его для нулевой
температуры. При ненулевой температуре возникает следующая функция:
\begin{equation}
J(x)=\int\limits_{-\infty}^x [I_{-\frac{1}{2}}(\xi)]^2d\xi.
\label{eq:ref_1_1_6}
\end{equation}
Её называют интегральной функцией Ферми-Дирака. Её свойства и способы вычисления также будут рассмотрены в этой книге. 
\\
