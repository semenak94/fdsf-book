
\S \textbf{4.1. Проблема трудоемкости квадратур}

Прямое вычисление функций ФД нецелого индекса требует применения
квадратурных формул к интегралу (1.5). При произвольных индексах $k$ (в том числе и целых!) такое интегрирование для получения высокой точности (16 верных десятичных знаков) требует очень подробной сетки ( $10^5 - 10^6$ узлов ) и является чрезмерно трудоемким. Однако для полуцелых индексов $k$
возможно кардинальное уменьшение трудоемкости при специальном
преобразовании подынтегрального выражения. Можно построить квадратуры с
экспоненциальной, то есть очень быстрой сходимостью.

Квадратуры с экспоненциальной сходимостью были предложены и исследованы в [ссылки на работу 2014 года, наши работы]. В [ссылка на статью SIAM] было рассмотрено интегрирование периодической функции $u(x)$ на ее периоде $[0,2\pi]$ по формуле трапеций на равномерной сетке с N интервалами. Была доказана следующая

\textbf{Теорема 1.} Пусть $u(x)$ есть функция, аналитическая в полуплоскости $Imx \geqslant -a, a > 0$. Тогда справедлива следующая мажорантная оценка погрешности формулы трапеций:
\begin{equation}
|\delta| \leqslant \frac{2\pi M}{[exp(aN)-1]},
\label{eq:ref4_1}
\end{equation}
где $N=max|u(x)|$ на отрезке интегрирования. TODO:добавить жирный кружок в конце Th.
Дадим комментарий. Неаналитичность функции $u(x)$ ниже указанной полуплоскости означает, что налинии $Im(x)=-a$ лежит по меньшей мере одна особая точка функции. Тогда, в силу периодичности $u(x)$, эти особые точки будут расположены на линии $Imx=-a$ также с периодом $2\pi$. Тогда, одна из этих особых точек будет лежать непосредственно под вещественным отрезком интегрирования. Тем самым, $a$ есть расстояние от отрезка интегрирования до ближайшей особой точки.

Интуитивно ясно, что теорему можно усилить. По-видимому достаточно требовать аналитичности $u(x)$ в полосе  $-a \leqslant Im x \leqslant a$. Кроме того, для $u(x)=const$, погрешность квадратур трапеций равна 0. Это значит, что вычитая из $u(x)$ константу, мы не меняем фактической погрешности. Поэтому оценку постоянной $M$ можно немного улучшить. Например, $если u(x)$ вещественно на вещественной оси, то $M=|$max$u+$min$u|/2$ на вещественном отрезке интегрирования. 

Экспоненциальная сходимость гораздо быстрее степенной. Поэтому такие квадратуры могут обеспечить экономичность расчетов. Покажем, как можно построить экспоненциально сходящиеся квадратуры для функций ФД полуцелых индексов.

%TODO вставить рисунок и пояснения к нему

\S \textbf{4.2. Сходимость квадратур}

Рассмотрим задачу вычисления интегралов от
функций $u(x)$, имеющих сколь угодно высокие непрерывные производные на
отрезке интегрирования $[a,b]$ (см. напр. [3-5]). Чаще всего на практике берут
равномерные или сводящиеся к равномерным сетки $ \omega_N={x_n,n=0,...,N}$ и
используют простейшие квадратурные формулы трапеций, средних, Симпсона
и т.п. Погрешность подобных формул имеет оценку
$const$, где $p$ есть порядок точности формулы, $h$ – шаг интегрирования, $TODO$ Такая сходимость называется
\textbf{степенной}, поскольку погрешность выражается через степень шага. Она
достаточно медленная, и для получения высокой точности требуется большое N .
Такие квадратуры довольно трудоемки.

Квадратуры Гаусса-Кристоффеля дают гораздо более быструю
сходимость. Например, классическая формула Гаусса для интегрирования на
отрезке $[-1,1]$ с весом $\rho(x)=1$ имеет погрешность (после упрощения
факториальных множителей)
\begin{equation}
\delta  \leqslant \sqrt{\frac{\pi}{N}}\frac{b-a}{4}\Bigg(e\frac{b-a}{8N}\Bigg)^{2N}M_{2N}.
\end{equation}
Квадратура Эрмита для отрезка $[-1,1]$ с весом $\rho(x)=(1-x^2)^{-1/2}$ имеет
погрешность
\begin{equation}
\delta \leqslant \sqrt{\frac{\pi}{N}}\Bigg(\frac{e}{2\sqrt{2}N}\Bigg)^{2N}M_{2N}.
\end{equation}
Погрешности (11-12) с точностью до логарифмически малых членов можно
записать в следующем виде:
\begin{equation}
\delta \sim \alpha \exp{(-\beta N)}.
\end{equation}
Зависимость от числа узлов является не степенной, а экспоненциальной,
поэтому такую сходимость будем называть \textbf{экспоненциальной}.

Заметим, что формулы со степенной сходимостью порядка $p$ требуют существования непрерывной производной лишь $p-$ого порядка подынтегральной функции независимо от числа узлов сетки $N$. Формулы Гаусса-Кристоффеля с $N$ узлами требуют существования $2N-$й непрерывной производной, т.е. при увеличении $N$ надо соответственно повышать гладкость функции.

Трудоемкость формул Гаусса-Кристоффеля несравненно меньше, чем у квадратур со
степенной сходимостью. Однако узлы и веса этих квадратур
найдены лишь для отдельных отрезков и весов $\rho(x)$ интегрирования. При этом
только для квадратур Эрмита эти веса и узлы найдены в виде простых формул
для произвольных $N$. Для остальных случаев узлы и веса точно вычисляются
(через радикалы) лишь для $N \leqslant 3$ или $N = 5$. Это сильно ограничивает
возможности практического использования таких квадратур.

Далее покажем, что если $u(x)$ четно продолжается через обе границы
отрезка, то формула трапеций на равномерной сетке дает экспоненциальную
сходимость. При этом коэффициент $\beta$ в экспоненте определяется расстоянием
до ближайшей особой точки в комплексной плоскости. Это открывает новые
возможности для построения квадратур малой трудоемкости.

\S \textbf{4.3. Случай экспоненциальной сходимости}

Пусть $u^{(p)}(x)$ существуют и непрерывны на $[a;b]$ при любых $p$. Требуется вычислить интеграл
\begin{equation}
%TODO
\end{equation}
Введем равномерную сетку $TODO$ и воспользуемся формулой
Эйлера–Маклорена, базирующейся на формуле трапеций [6,7]:
\begin{equation}
%TODO
\end{equation}
Если оборвать эту сумму на члене P , то первый отброшенный член будет
остаточным. Его величина есть $TODO$. В этом случае формула (15) имеет
степенную сходимость.

Пусть $u(x)$ такова, что все её нечётные производные на правой и левой
границах одинаковы: $u^(2p-1)(a)=u^(2p-1)(b)$. Тогда в (15) сумма обращается в
нуль. Оставшаяся часть квадратур является просто формулой трапеций. Из
этого следуют

%\begin{theorem}
\textbf{Утверждение 1.} Пусть подынтегральная функция $u(x)$ имеет сколь
угодно высокие производные, причем нечетные производные на правой и левой
границах одинаковы: $u^(2p-1)(a)=u^(2p-1)(b)$ . Тогда формула трапеций на
равномерной сетке имеет сходимость выше степенной.
%\end{theorem}
%\begin{theorem}
\textbf{Частный случай.} Утверждение 1 справедливо, если $u(x)$ чётно
продолжается через обе границы отрезка: $u^(2p-1)(a)=u^(2p-1)(b)=0$.
%\end{theorem}

Таким образом, установлен класс функций, для которого формула
трапеций имеет сверхстепенную сходимость. Остается найти закон этой
сходимости. Проведем ее изучение на следующем тестовом примере:
\begin{equation}
U(q,r,c) = \int_0^{\pi} \frac{(c^2-1)c^r\cos (rx)}{(c^2-2c\cos x+1)^q}dx,c>1.
\end{equation}
Параметры $r \geqslant 0$, $q \geqslant 1$ берутся целыми. Тогда подынтегральное выражение
четно на обеих границах отрезка, его нечетные производные на границах
обращаются в нуль, и пример удовлетворяет требованиям Утверждения 1. При
$q = 1$ известно точное значение интеграла [8]:
\begin{equation}
U(1,r,c) = \pi.
\end{equation}
При $q!=1$ интеграл (16) не выражается через элементарные функции от
параметров.

Для тщательного численного выявления закономерностей все расчеты
проводились с повышенной разрядностью ( 45 десятичных знаков ) с помощью
библиотеки языка С++ boost::multiprecision.

Расчеты интеграла (16) при фиксированных параметрах проводились на
сетках с разным числом интервалов $N$. Погрешность расчетов при $q =1$
определялась непосредственным сравнением с точным ответом (17). На рис.1
показана зависимость погрешности от $N$ при $r = 0$ и различных значениях с в
полулогарифмическом масштабе. Каждому значению с соответствует своя
линия погрешности. Видно, что при всех значениях с кривые погрешности в
этом масштабе являются прямыми. Это означает, что погрешность подчиняется
закону
\begin{equation}
ln \delta_N = \alpha - \beta N, \beta = const lnc.
\end{equation}
При других значениях параметров картина была аналогичной. На рис.2 показан
случай $q =1,r = 2$. Опять линии погрешности являются прямыми.

Для $q >1$ точный ответ неизвестен. В этом случае для получения значений
погрешности можно воспользоваться следующими соображениями. При
закономерности (18) разности значений $U$ при возрастании $N$ на единицу
также должны ложиться на прямую в полулогарифмическом масштабе (это
напоминает апостериорную оценку погрешности по методу Ричардсона для
квадратур со степенной сходимостью). На рис. 3 приведены графики таких
разностей для $q = 2,r = 1$. Они также оказываются прямыми. Все это позволяет
сделать эвристическое

%\begin{theorem}
\textbf{Утверждение 2.} При выполнении условий Утверждения 1 погрешность
формулы трапеций экспоненциально зависит от числа узлов сетки $N$.
%\end{theorem}

Попробуем выяснить, от чего зависит коэффициент $\beta$ в (18). Он не
должен зависеть от максимумов модулей каких-либо производных $u(x)$ ,
поскольку они входят в суммы формул Эйлера-Маклорена (15) и приводят к
степенной сходимости. Поэтому рассмотрим гипотезу о связи $\beta$ с особыми точками
подынтегрального выражения.

Если скобка в знаменателе подынтегрального выражения в (номер формулы теста) обращается в нуль, то подынтегральное выражение имеет полюс порядка $q$. Это происходит при
\begin{equation}
\frac{c^{2} + 1}{2c}=cos(x)=\frac{e^{ix}+e^{-ix}}{2}.
\end{equation}
Это уравнение имеет два решения: $e^{ix}=c$ или $e^{ix}=1/c$. Следовательно, имеется две цепочки полюсов кратности q в точках
\begin{equation}
x^{\star} = 2\pi m \pm iln(c),-\infty \leqslant m \leqslant +\infty.
\end{equation}
%TODO сделать рисунок для полюсов тестового примера
Из рис № видно, что наименьшее расстояние между каким-либо из полюсов и ближайшей к нему
точкой отрезка интегрирования есть $ln(c)$.

Рис.1. Погрешность квадратуры трапеций для (18) при $p = 0$ и $q =1$.
Цифры около линий – величины с .
Рис.2. Погрешность квадратуры трапеций для (18) при $p = 2$ и $q =1$.
Цифры около линий – величины с .
Рис.3. Погрешность квадратуры трапеций для (16) при $p =1$ и $q = 2$.
Цифры около линий – величины с .

Предварительный просмотр графиков показал, что наклон $\beta ~ ln(c)$. Для
тщательного анализа на рис.4 показано отношение $\beta / ln(c)$ в зависимости от $с$
для нескольких значений $q =1,2$ и $r = 0,1,2$. Видно, что для полюса первого
порядка ( $q =1$ ) при $r=0$ это отношение с высокой точностью не зависит от
$с$ и равно $2$. При $r=1$ это отношение равно 2 при $ln(c) \approx 0$; при увеличении с это отношение несколько уменьшается, причем линия в пределах графика блзка к прямой. При $r=2$ линия также начинается в точке 2, но её наклон ещё немного увеличивается.

Для полюсов второго порядка ( $q = 2$ ) линии для разных $r$ начинаются не со значения $2$, а с несколько меньшего значения $1.93$. Наклоны линий с разными $r$ также несколько больше, причем даже линия c $r=0$ уже имеет наклон. Наименьшее отношение в пределах указанного графика $1.70$.

Рис.4. Зависимость $\beta / ln(c)$ от величины с для различных $p$ и $q$ .
Это позволяет сделать

%\begin{theorem}
\textbf{Утверждение 3.} Наклон $\beta$ в (18) с хорошей точностью пропорционален
расстоянию от отрезка интегрирования до ближайшего полюса интегрируемой
функции в комплексной плоскости. Коэффициент пропорциональности достигает 2 в случае полюса первого порядка, и несколько уменьшается с увеличением кратности полюса и расстояния до него.
%\end{theorem}

\textbf{Практические рекомендации.} 1. При использовании квадратурных
формул со степенной сходимостью удобно сгущать сетки по $N$
последовательно вдвое. Это позволяет использовать обычную процедуру
Ричардсона для получения априорной асимптотически точной оценки
погрешности. Такое сгущение экономично, поскольку суммарный объем всех
расчетов лишь вдвое превышает объем расчетов на последней сетке [4].

Для квадратур с экспоненциальной сходимостью (18) также можно
пользоваться процедурой Ричардсона, если сгущать сетки не вдвое, а каждый
раз увеличивая $N$ на 1. При этом будет получаться асимптотически точная
апостериорная оценка погрешности. Однако такое сгущение сеток
экономически невыгодно, поскольку суммарный объем вычислений будет в
$\sim N / 2$ раз больше, чем расчет на последней сетке.

Поэтому в практических расчетах удобнее увеличивать $N$ в 2 раза. Из (18)
нетрудно получить, что при этом $\delta_{2N} \sim \delta_N^2$. Такой закон убывания напоминает
сходимость ньютоновских итераций вблизи простого корня: число верных
десятичных знаков приблизительно удваивается с увеличением $N$ в 2 раза.
Поэтому на практике останавливаются на такой сетке $2N$ , когда отклонение от
результата на предыдущей сетке становится меньше $\varepsilon^{2/3}$ , где $\varepsilon$ – ошибка
единичного округления компьютера.

\textbf{2.} Для формулы трапеций на равномерной сетке полезен следующий
прием, вдвое уменьшающий трудоемкость вычислений. На сетке с $N$ узлами и
шагом h формула трапеций имеет вид
\begin{equation}
U_N=h(\frac{u_0}{2}+u_1+...+u_(N-1)+\frac{u_N}{2}).
\end{equation}
При удвоении сетки все узлы предыдущей сетки становятся четными узлами
новой сетки, и заново вычислять значения функций в них не надо. Достаточно
найти значения функции в новых ( нечетных ) узлах и вычислить
\begin{equation}
U_{2N}=\frac{1}{2}U_N + \frac{h}{2}(u_1+u_3+u_5...+u_{2N-1}).
\end{equation}
где нечетные индексы относятся к узлам новой сетки.


\S \textbf{4.4. Сравнение теории с эвристическими оценками}

Теорема 1 с оценкой погрешности (\ref{eq:ref4_1}) и эвристические Утверждения 1-3, полученные из численных экспериментов, качественно близки. Однако между ними имеется ряд различий. Обсудим их.

Теорема 1 строго доказана. В ней рассмотрен интеграл от периодической аналитической функции, взятой по полному периоду. Получена мажорантная оценка погрешности, справедливая для любого типа особых точек, включая существенно особые.

При получении эвристических Утверждений 1-3 также неявно предполагалась аналитичность функции. Однако периодичность функции не предполагалась. Правда, в рассмотренном тесте функция была периодической, но интеграл брался только по половине периода. Поэтому на самом деле требовалось лишь равенство нечетных производных на концах отрезка интегрирования (такое обобщение существенно для практических применений). Однако в численных экспериментах рассматривался только простейший случай особых точек - полюса первого и второго порядков. Это более благоприятная ситуация, позволившая получить более сильные оценки, причем асимптотически точные, а не мажорантные.

И Теорема 1, и Утверждения 1-3 дают экспоненциальную сходимость. Однако в Теореме 1 коэффициент перед числом интервалов $N$ равен расстоянию до ближайшей особой точки. В эвристических оценках этот коэффициент почти в 2 раза больше, то есть реальная сходимость гораздо быстрее теоретической оценки (\ref{eq:ref4_1}). Однако, при увеличении порядка полюса, этот коэффициент уменьшается. Поэтому возможно, что для существенно особых точек он уменьшится до теоретического значения.

В знаменателе теоретической оценки (\ref{eq:ref4_1}) из экспоненты вычитается 1. В эвристических оценках этого не наблюдалось. Скорее всего, вычитание 1 связано с особенностями теоретического вывода. Заметим, что это вычитание существенно лишь, когда особая точка стремится к отрезку интегрирования. Если расстояние до особой точки не очень мало, то при разумном числе интервалов N экспонента будет существенно превышать 1 и указанный эффект станет незаметным.


\S \textbf{4.5. Квадратуры для функций ФД полуцелых индексов}

%Выше были построены формулы, по которым можно рассчитывать функции полуцелого индексов при $x \leqslant 0$ или $x \geqslant x_{min}, x >>1$ с требуемой высокой точностью. В оставшемся промежутке $0 < x < x_{min}$ можно вычислять оставшиеся функции полуцелых индексов с помощью квадратур интеграла (1).
Сравнительно экономичным способом прямого вычисления функций ФД могут служить квадратуры с
экспоненциальной сходимостью. При этом возникает два различных случая,
которые опишем ниже.

\textbf{Функции индекса $k \geqslant -1/2$.} Они определяются через сходящийся
интеграл (1.5). Сделаем в интеграле (1) замену переменных $t=\tau^2$ . Тогда интеграл
(1) приведется к следующему виду
\begin{equation}
TODO
\end{equation}
При полуцелых $k \geqslant -1/2$ показатель степени в подынтегральном выражении
будет целым четным неотрицательным числом. Поэтому подынтегральное
выражение будет четной функцией $\tau$ , и все его нечетные производные на
нижнем пределе интегрирования $\tau = 0$ обращаются в нуль. Тем самым, на
нижнем пределе интегрирования удовлетворяется условие Частного случая
Утверждения 1.

На верхнем пределе интегрирования подынтегральное выражение убывает
как $exp(-\tau^2)$ . При этом все производные быстро стремятся к нулю. Но
применять формулу Эйлера-Маклорена на равномерной сетке (15) к
бесконечному интервалу невозможно. Поэтому для численного интегрирования
надо обрезать интеграл (30) и положить
\begin{equation}

\end{equation}
Верхний предел $T$ нужно выбирать так, чтобы во-первых, отброшенной частью
интеграла можно было бы пренебречь, во-вторых, чтобы производные при T
были бы настолько малы, чтобы их вклад в формулы Эйлера-Маклорена был
пренебрежимо мал. Тогда на отрезке $0 \leqslant \tau \leqslant T$ формула Эйлера-Маклорена на
равномерной сетке (15) обеспечит экспоненциальную сходимость.

Вообще говоря, оценка минимального $T$ зависит от $x$ и $k$. Очевидно,
$T$ возрастает при увеличении $x$ или увеличении $k$. Нетрудно оценить
отброшенную часть интеграла (31). Для не малых $x > 0$ она составляет
примерно  $T^{2k}exp(x-T^2)$ . Для получения относительной ошибки эту величину
нужно сравнить с асимптотикой интеграла (1.5) $x^{k+1}/(k+1)$. Для получения
относительной погрешности $\varepsilon$ должно выполняться условие
\begin{equation}
TODO
\end{equation}
целесообразно брать $T$ "с запасом" для обеспечения надежности алгоритма. Но
левая часть (32) быстро возрастает при увеличении $T$, так что на практике
достаточно умеренного увеличения $T$.

Если определять T по значению min x , то такая величина будет пригодна
для всех x  xmin . Рассмотрим два крайних случая при $10^{-16}$ (double
precision). Первый соответствует min k = -1/2, x = 39 (см. Таблицу 4); он дает
T = 8.4. Второй соответствует min k = 7/2, x = 29; он дает T > 8.5. Для
остальных функций полуцелых индексов k  -1/ 2 получаются практически
такие же результаты. Поэтому для всех индексов $k \geqslant -1/2$ и для любых
аргументов min x  x далее будем единообразно брать T=12. Это создает
необходимый запас надежности, включая обеспечение малости высоких
производных на правой границе.

Подробнее обсудим сходимость квадратур. На рис. 5 изображена
комплексная плоскость переменной интегрирования $\tau$. Отрезок
интегрирования выделен жирной линией – это вещественная положительная
полуось. Показатель экспоненциальной сходимости определяется расстоянием
от ближайшего полюса до промежутка интегрирования. Найдем положение полюсов. Это полюсы первого порядка, возникающие при обращении в нуль знаменателя подынтегрального выражения. Это происходит при $exp(\tau^2 -x) = -1$. Отсюда получается цепочка полюсов:
\begin{equation}
\tau_m^2 = x + i\pi(1+2m), -\infty \leqslant m \leqslant +\infty, -\infty \leqslant x \leqslant +\infty.
\end{equation}
Извлечение квадратного корня из правой части дает
\begin{equation}
Re \tau_m = \sqrt{\frac{x+\sqrt{x^2 + \pi^2(1+2m)^2}}{2}},
Im \tau_m = \sqrt{\frac{-x+\sqrt{x^2 + \pi^2(1+2m)^2}}{2}}
\end{equation}
Ближайший к промежутку интегрирования по $\tau$ полюс соответствует $m=0$. Его расстояние от промежутка интегрирования равно
\begin{equation}
Im \tau_0 = \sqrt{\frac{-x+\sqrt{x^2 + \pi^2}}{2}} = \pi \sqrt{2(x+\sqrt{x^2+\pi^2})}, -\infty \leqslant x \leqslant +\infty.
\end{equation}
При $x< 0$ удобнее пользоваться первым выражением, a при $x>0$ - вторым. Видно, что 
\begin{equation}
Im \tau_0 \to \sqrt{|x|}, x->-\infty
Im \tau_0 = \sqrt{\pi / 2}, x=0
Im \tau_0 \to  pi /(2\sqrt{x}), x->+\infty
\end{equation}
Поведение зависимости Im$\tau(x)$ изображено на рисунке. % Вставить рисунок
Таким образом, при $x \to -\infty$ расстояние до ближайшего полюса быстро возрастает, и формула трапеций обеспечивает высокую точность при небольшом числе узлов. Напротив, при $x \to +\infty$ расстояние до ближайшего полюса быстро уменьшается, и сходимость будет существенно медленнее. Однако даже в этом случае число узлов остается умеренным благодаря экспоненциальной скорости сходимости.


Рис.5. Полюса подынтегрального выражения для функций ФД.

Численные расчеты подтверждают эти соображения. Интеграл (31)
вычисляется на отрезке [0,12] по формуле трапеций (15). Вычисление
проводится с автоматическим сгущением сетки до выхода на ошибки
округления. Эта процедура аналогична тому, что изложено в Практических
рекомендациях в Разделе 4.4. При этом окончательные сетки содержат N = 48
или 96 узлов при x = 0 и доходят до N =192 при $x=30-40$ . Действительно
видно, что число узлов слабо зависит от $x$ при широких пределах изменения $x$.
% TODO
\textbf{Индекс k = -3/2.} Для этого индекса интеграл (1) оказывается
расходящимся. Поэтому для вычисления функций целесообразно
переопределить ее через производную функции старшего индекса и провести
следующее преобразование [9]
\begin{equation}
TODO
\end{equation}
В последнем выражении сделаем замену переменных 2 t  и одновременно
обрежем интеграл по верхнему пределу, аналогично показанному выше.
Получим выражение
\begin{equation}
TODO
\end{equation}
Подынтегральное выражение есть четная функция  , так что при достаточно
большом T квадратура трапеций на равномерной сетке для интеграла (34)
будет сходиться экспоненциально. Оценку T производим аналогичным
образом. Приравнивая отброшенную часть интеграла к  -й доле главного члена
асимптотики, получаем минимальную оценку
\begin{equation}
TODO
\end{equation}
Подставляя сюда наихудшее условие min x  x  44 (см. Таблицу 4), находим
T  9 . Для создания достаточного запаса надежности опять положим T 12.
Тогда квадратура трапеций с контролем экспоненциальной сходимости
обеспечивает точность double при N  96 узлах сетки для x  0 и N  384 при
min x  x .
