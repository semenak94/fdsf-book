\chapter{АППРОКСИМАЦИИ ФУНКЦИЙ ФД}

\S \textbf{6.1. Глобальные аппроксимации}
\\

\textbf{Обобщенные формулы.} Для несложных оценочных расчетов физикам полезно
иметь простые формулы, непрерывно и гладко описывающие функции ФД во всем диапазоне значений аргумента
$-\infty \leqslant x \leqslant +\infty$. Для успешного применения такие формулы должны описывать главные
члены левых и правых асимптотик функций. Физически это обеспечивает переход фермионов в
идеальный газ при высоких температурах, и в полностью вырожденный газ при низких
температурах.

Для построения таких формул учтем следующие соображения. Левые асимптотики ($x \to -\infty$) функций ФД являются разложениями по $e^x$. Правые асимптотики ($x \to +\infty$) имеют главный член $\sim x^{k+1}$ и содержат разложения по степеням $x^{-2}$. Эти разложения сильно различаются по своей математической структуре. Левая асимптотика грубо искажает поведение функции ФД при $x > 0$, а правая при $x < 0$, поэтому построение оценочных формул в виде явной зависимости от $x$ нетривиально.

Однако левые асимптотики всех функций ФД качественно сходны между собой, и правые асимптотики также качественно сходны между собой. Поэтому естественна идея выражать функции ФД индекса $k$ через функции ФД индекса $m$. Такие формулы были предложены в [Ритус ЖВМ,препринт]. Рассмотрим их построение. При этом учтем, что когда аргумент $x$ меняется от $-\infty$ до $+\infty$, значения функции ФД (при $k>-1$) монотонно возрастают от $0$ до $+\infty$. При этом качественное поведение функций при $x \to +\infty$ и $x \to -\infty$ качественно различается.

При $x \to -\infty$ все функции ФД разлагаются в степенные ряды по $e^x$, причем главным членом является первая степень $e^x$ (TODO ссылка на формулу). Это означет, что каждая функция разлагается в степенной ряд по другой функции. Из () получаем следующее разложение:
\begin{equation}
I_k = \Gamma(k+1) \sum\limits_{p=0}^{\infty} a_p[\frac{I_m}{\Gamma(m+1)}]^{1+p}.
\label{eq:ref_6_1_1}
\end{equation} 
Приведем главный и 2 следущих коэфициента этого разложения:
\begin{equation}
a_0=1, TODO.
\label{eq:ref_6_1_2}
\end{equation} 
Этот ряд сходится при достаточно малых $I_m$; точное установление границ сходимости не производилось.

При $x \to +\infty$ главный член функции ФД имеет вид некоторой степени $x$. Он домножается на асимптотическое разложение по степеням $x^{-2}$. Тогда из (TODO ссылка нф) вытекает следующее разложение:
\begin{equation}
I_k \approx \frac{1}{k+1} \sum\limits_{p=0} b_p[(m+1)I_m]^{\frac{k+1-2m}{m+1}}.
\label{eq:ref_6_1_3}
\end{equation} 
Его главный и 2 следущих коэфициента равны
\begin{equation}
b_0=1, TODO.
\label{eq:ref_6_1_4}
\end{equation} 
Это разложение является асимптотическим.

Построим следующую вполне гладкую аппроксимацию, выражающую одну функцию ФД через другую и правильно передающую главные члены и параметры малости обоих разложений ($\ref{eq:ref_6_1_1}$) и ($\ref{eq:ref_6_1_3}$):
\begin{equation}
ЕTODO
\label{eq:ref_6_1_5}
\end{equation}
Здесь первая сумма обеспечивает качественно правильное разложение при $I_m \to 0$, а вторая сумма делает то же при $I_m \to +\infty$. Заметим, что последний член первой суммы является её естественным окончанием. Но показатели степени во второй сумме подобраны так, что он может формально рассматриваться как член второй суммы. Это должно улучшить "сшивание" двух качественно отличающихся разложений. 

Значения свободных параметров $c_p,1 \leqslant p \leqslant N-1$, и числа членов в суммах можно варьировать для удовлетворения различых критериев. Например, соответствующим выбором младших коэффициентов $c_k,1 \leqslant p \leqslant P_0 \leqslant P,$ можно добиться точной передачи $P_0$ членов ряда ($\ref{eq:ref_6_1_1}$), следующих за главным. Аналогично можно выбрать старшие коэффициенты $c_k,N-1 \geqslant p \geqslant P_1 \geqslant P$, так, чтобы правильно передать $N-P_1$ членов разложения ($\ref{eq:ref_6_1_3}$). Остальные коэффициенты естественно определяются из условия минимума погрешности в $\| \cdot \|_C$ или $\| \cdot \|_{L_2}$, при этом для положительных и сильно меняющихся по величине функций ФД целесообразно минимизировать не абсолютную, а относительную погрешность.

Очевидно, увеличение $N$ позволяет уменьшить погрешность аппроксимации, причем трудоемкость вычислений возрастает не сильно, ибо основное время расходуется на вычисление не сумм, а двух нецелых степеней. Значение $P$ разумно полагать близким к $n/2$.

\textbf{\textit{Замечания. $1^{\circ}$}} В задачах атомной физики аргумент $x$ играет роль потенциала поля, в котором движется электрон, а функция $I_{\frac{1}{2}}(x)$ пропорциональна плотности электронов в соответствующей точке поля. Поэтому наибольшее практическое значение имеют формулы, выражающие функции ФД либо через $I_0(x)$ (в этом случае устанавливается зависимость от потенциала поля), как сделано выше, либо через $I_{\frac{1}{2}}(x)$, т.е. через плотность электронов.

\textbf{$2^{\circ}$.} Формулу ($\ref{eq:ref_6_1_5}$) затруднительно применять для получения прецизионных аппроксимаций. Прецизионные аппроксимации требуют большого числа коэффициентов. При этом задача определения методом наименьших квадратов или другим методом становится плохо обусловленной. Кроме того, при большом числе членов будет очень мал показатель степени $\kappa$, что также ухудшает обусловленность задачи. Однако для построения гладких оценочных аппроксимаций невысокой точности с небольшим числом членов формула ($\ref{eq:ref_6_1_5}$) удобна.

\textbf{$3^{\circ}$.} Вычисление $I_k(x)$ через $I_m(x)$, где в свою очередь $I_m(x)$ непосрежственно вычисляется по $x$, означает введение промежуточного аргумента. Наиболее удобным оказался выбор $m=0$, потому что тогда промежуточный аргумент явно выражающаяся через элементарные функции от $х$:
\begin{equation}
y(x) \equiv I_0(x) = ln(1+e^x); 0 \leqslant y < +\infty, -\infty < x < +\infty.
\label{eq:ref_6_1_6}
\end{equation}
Такой аргумент ведет себя как $e^x$ при $x \to -\infty$, и как $x$ при $x \to +\infty$. Таким образом, он гладко склеивает экспоненту левой асимптотики с полиномом правой асимптотики. Поэтому целесообразно искать аппроксимацию $I_k(x)$ в виде некоторых явных зависимостей от промежуточного аргумента $y$.

Ниже построен большой набор таких приближенных формул.
\\

\textbf{Двучленные формулы.} Двучленная формула была предложена в [8]. Запишем ее в несколько более удобной форме:
\begin{equation}
I_k(x) \approx \frac{y}{k+1}(\Gamma(k+2)^{1/k} + y)^k, 0 \leqslant y < +\infty.
\label{eq:ref_6_1_7}
\end{equation}
Эта формула не содержит подгоночных параметров, её можно применять для любых индексов $k > -2$.

% TODO (перестроить рисунок для всех индексов )
На рис.№. показаны погрешности двучленной формулы ($\ref{eq:ref_6_1_7}$) для целых индексов $k$. Видно, что для $k > 0$ эта формула всюду
завышает значение функции. Погрешность её значительна. Максимальное завышение
составляет $30 \% $ для $k = 1$ и доходит до $190 \%$ для $k = 4$. Поэтому двучленная формула пригодна
только для очень грубых оценок.
% TODO подкорреттировать текст после перестроения графика
\\

\textbf{Трехчленная формула} была предложена в [9-10]. Ее также запишем в несколько
удобной форме:
\begin{equation}
I_k(x) \approx \frac{y}{k+1} \bigg( [\Gamma(k+2)]^{\frac{3}{k}}(1+cy) + y^3 \bigg)^{\frac{k}{3}};
\label{eq:ref_6_1_8}
\end{equation}
коэффициент $c$ подбирается так, чтобы минимизировать относительную погрешность
аппроксимации. Формула построена так, что при $y \to 0$ она точно передает главный
член левой асимптотики и порядок малости следующего члена разложения (но не точный
 коэффициент в этом члене). При $y \to +\infty$ она точно передает главный член асимптотики и порядок малости следующего члена. Это улучшает качественное поведение
аппроксимации.

Формула ($\ref{eq:ref_6_1_8}$) оказалась несравненно лучше двучленной формулы. Наличие подгоночного параметра $c$ 
позволяет сделать профиль погрешности знакопеременным. Подбирая $с$ из условия чебы
шевского альтернанса, мы минимизируем относительную погрешность. Соответствующие оптимальные
значения $с$ приведены в табл.2; в ней приведены значения для целых индексов $k$,
взятых из [TODO ссылка на уточнение прецизионных аппроксимаций], и полуцелых индек
сов $k$, взятых из [Ритус ЖВМб препринт](для k=-1/2, k=1/2, k=3/2) и [master Диплом](k=3/2, 5/2, 7/2). Видно, что коэффициенты $с$ монотонно убывают с ростом $k$. В
табл.2 приведены также погрешности полученных формул в процентах; они тем больше,
чем сильнее отличается $k$ от нуля. Эти погрешности много меньше, чем для формулы
($\ref{eq:ref_6_1_7}$), и даже для $k=4$ не превышают $6.4\%$. Видно, что трехчленную формулу ($\ref{eq:ref_6_1_8}$) уже можно рекомендовать для удовлетворительных оценочных расчетов. 

\begin{table}[]
\caption{\textbf{Коэффициенты и погрешности формулы ($\ref{eq:ref_6_1_8}$).}}
\begin{center}
\begin{tabular}{|c|c|c|}
\hline
$k$ & $c$ & $\Delta_{max} (\%)$ \\
\hline
-3/2 & & \\
-1/2 & 1.62 & 0.7 \\
 1/2 & 1.18 & 0.8 \\
 1   & 1.01 & 1.6 \\
 3/2 & 0.87 & 2.6 \\
  2  & 0.77 & 3.2 \\
 5/2 & 0.67 & 4.0 \\
  3  & 0.60 & 4.8 \\
 7/2 & 0.54 & 5.6 \\
  4  & 0.48 & 6.4 \\
\hline
\end{tabular}
\end{center}
\end{table}

Типичный график относительной погрешности этой формулы для для $k = 2$ приведен на рис№. Видно, что погрешность знакопеременна, имеет 2 равных по модулю экстремума, т.е удовлетворяет условию чебышевского альтернанса, и стремится к нулю на бесконечности.
% TODO попробовать поместить все кривые погрешности на один рисунок, посомтреть, как будет выглядеть
\\

\textbf{Обобщенные трехчленные формулы.} Запишем частный случай обобщенных формул (ссылка на номер) с тремя членами, правильно передающие главные члены левой и правой асимптотик:
\begin{equation}
\begin{aligned}
&I_k = \alpha_0 I_m \cdot (1 + \alpha_1 I_m + \alpha_2 I_m^{\eta})^\kappa,\\
&\alpha_0 = \frac{\Gamma(k+1)}{\Gamma(m+1)},\\
&\alpha_2 = \Bigg[\frac{\Gamma(m+1)}{\Gamma(k+1)} \frac{(m+1)^{\frac{k+1}{m+1}}}{k+1}\Bigg]^{1/{\kappa}}, \\
&\kappa = \frac{k - m}{m + 3}, \\
&\eta = \frac{(m+3)}{m+1}.
\end{aligned}
\label{eq:ref_6_1_9}
\end{equation}
Коэффициент $\alpha_1$ подбирается для минимизации относительной погрешности. Погрешность знакопеременна и имееет два экстремума, удовлетворяющих условию чебышевского альтернанса.

Возьмем $m=1/2$, что означает выбор электронной плотности в качестве аргумента. В таблице № приведены значения всех коэффициентов и погрешностей формулы ($\ref{eq:ref_6_1_9}$) для индексов $k = -1/2, 0, 3/2$. Этот набор индексов обеспечивает расчеты статистической модели атома Томаса-Ферми; в частности, значение $k=3/2$ дает выражение давления или энергии однородного электронного газа через его плотность. Видно, что достигается превосходная точность аппроксимации $0.6\%$. Этого достаточно для многих приложений, ибо неточность физических моделей зачастую превосходит эту величину.

Заметим, что такая точность при единственном свободном параметре свидетельствует об удачном выборе вида аппроксимации.
\\

\textbf{Пятичленные формулы} дают еще более хорошие результаты. Запишем их так, чтобы
члены с первого по третий выглядели как разложение по степеням $y \equiv I_0$ при $y \to 0$, а
члены с третьего по пятый – как разложение по степеням $y^{-2}$ при $y \to +\infty$:
\begin{equation}
I_k(x) \approx \frac{y}{k+1}(\Gamma(k+2)^{\frac{6}{k}}(1 + c_1y + c_2y^2) + c_3y^4 + y^6)^{\frac{k}{6}}.
\label{eq:ref_6_1_10}
\end{equation}
Выбирая коэффициенты из различных соображений, рассмотрим три варианта этой
формулы.

\textbf{\textit{$1^{\circ}$. Улучшенные асимптотики.}} Выберем коэффициенты $c_1$ и $c_3$ так, чтобы правильно передать вторые члены левой и правой асимптотик:
\begin{equation}
с_1 = 3 (1-2^{-k})/k,
c_3 = \pi^2 (k + 1);
\label{eq:ref_6_1_11}
\end{equation}
коэффициент $c_2$ оставим в качестве подгоночного. Такая формула обеспечивает описание
 не только пределов идеального и вырожденного ферми-газа, но и ближайшей поправки при
уменьшении идеальности либо снятии вырождения. Особенно важен коэффициент $c_3$, поскольку он позволяет описать тепловые свойства почти вырожденного газа (например, теплоемкость или проводимость при малых температурах).

Один подгоночный коэффициент $c_2$ может обеспечить лишь один нуль погрешности.
Подбираем $c_2$ из условия чебышевского альтернанса. Это позволяет минимизировать погрешность.
В табл.3 приведены оптимальные значения коэффициентов $c_2$ для целых и полуцелых значений $k$, а также погрешности полученных формул в процентах. Полученные погрешности составляют не более $2\%$ даже для больших $k$, что втрое лучше, чем для формулы (7). Тем самым эта формула предпочтительнее для оценочных расчетов.

\textbf{\textit{$2^{\circ}$. Низкотемпературная асимптотика.}} Передача второго члена асимптотики при высоких температурах обычно не
столь важна. Поэтому можно коэффициенты $с_1$ и $с_2$ сделать свободными параметрами,
а коэффициент $c_3$ сохранить согласно ($\ref{eq:ref_6_1_11}$). Это позволяет ввести второй нуль в график
погрешности, а коэффициенты $с_1$ и $с_2$ подобрать из условия чебышевского альтернанса.
Значения этих коэффициентов и соответствующие им значения максимальных погрешностей приведены в табл.4. Видно, что при этом относительная точность не хуже $1\%$. Это вдвое лучше, чем в табл.3.
% Графики
% TODO
Таблица 3. Коэффициенты и погрешности
формулы (8), вариант а).
Таблица 4. Коэффициенты и погрешности
формулы (8), вариант б) .
%k c2 dmax (%) k c1 c2 dmax (%)
%1 1.73 0.9 1 1.15 1.99 0.45
%2 0.98 1.3 2 0.93 1.11 0.60
%3 0.62 1.8 3 0.75 0.69 0.70
%4 0.42 1.9 4 0.60 0.47 0.80

\textbf{\textit{$3^{\circ}$. Наилучшая точность.}} Пусть важна минимальная погрешность, а вторым членом правой асимптотики можно пожертвовать. Тогда все три коэффициента $с_1 , с_2 , с_3$ можно использовать как подгоночные и выбирать из условия чебышевского альтернанса. График погрешности
при этом будет иметь три нуля. Соответствующие значения коэффициентов и погрешностей приведены в табл.№. Видно, что погрешности еще вдвое уменьшаются по сравнению с табл. 4 и не превышают $0.5 \%$, а для функций с небольшими инжексами составляют $\sim 0.2\%$. Это превосходная точность для столь простых формул.
\begin{table}[]
\caption{\textbf{Коэффициенты и погрешности формулы ($\ref{eq:ref_6_1_11}$),вариант в).}}
\begin{center}
\begin{tabular}{|c|c|c|c|c|}
\hline
$k$ & $c_1$ & $c_2$ & $c_3$ & $\delta_{max} (\%$) \\
\hline
-3/2 & & & \\
-1/2 & 1.846 & 5.430 & 7.166 & 0.28 \\
 1/2 & 1.44  & 2.47  & 16.58 & 0.14 \\
  1  & 1.28  & 1.78  & 21.50 & 0.20 \\
 3/2 & 1.14  & 1.32  & 26.60 & 0.28 \\
  2  & 0.99  & 1.02  & 31.42 & 0.30 \\
 5/2 & 0.87  & 0.801 & 36.513 & 0.41 \\
  3  & 0.78  & 0.65  & 41.45 & 0.45 \\
 7/2 & 0.70  & 0.53  & 46.430 & 0.47 \\
  4  & 0.63  & 0.44  & 51.59 & 0.50 \\
\hline
\end{tabular}
\end{center}
\end{table}

Отметим, что подобранные значения $с_3$ оказались близкими к теоретическим значениям
($\ref{eq:ref_6_1_11}$). Во-первых, это свидетельствует об удачном выборе аппроксимации. Во-вторых, это означает, что полученными фомулами можно удовлетворительно пользоваться для описания теплоемкости и других аналогичных свойств почти вырожденного электронного газа.
\\

\S \textbf{6.2. Прецезионные аппроксимации для целых $k$ при $x \leqslant 0$}
\\

\textbf{Вид аппроксимации.} Ранее говорилось, что значение функций целого индекса при $x > 0$ выражается через значение этой функции при $x < 0$ при помощи несложной формулы (). Поэтому нужно иметь способы прецизионного вычисления $I_k(x)$ при целых $k$ для $x \geqslant 0$. 

Такими прецизионными формулами могут служить всюду сходящиеся ряды (). Однако для получения относительной погрешности $\varepsilon=10^{-16}$ при $x \sim 0$ требуется суммировать $~40$ членов ряда. Это может оказаться недостаточно экономичным для стандартных программ. Поэтому рассмотрим, как строить более экономичные прецизионные формулы.

Используем следующие соображения. При $x \to -\infty$ главный член асимптотики есть $I_k(x) \approx \Gamma(k+1) e^x$.
При x асимптотика имеет не экспоненциальное, а степенное поведение: $I_k(x) \approx x^{k+1}/(k+1)$. Поэтому попробуем взять в качестве аргумента $y = I_0(x)$.

Аппроксимация многочленом редко бывает удачной, хотя она широко
используется в литературе. Обычно лучшие результаты дает рациональная
аппроксимация, то есть приближение отношением многочленов. Такая аппроксимация может передать разные асимптотики функций. В данном случае мы выбрали следующую аппроксимацию:
\begin{equation}
I_k(x) \approx \Gamma(k+1)y \Bigg( \frac{\sum\limits_{n=0}^{N+1} a_ny^{n}}{\sum\limits_{n=0}^{N} b_ny^{n}} \Bigg)^k,a_0=1,b_0=1,x \leqslant 0.
\label{eq:ref_6_2_1}
\end{equation}
При $x \to -\infty$ аппроксимация ($\ref{eq:ref_6_2_1}$) точно передает первый член ряда (4). Если
провести разложение ($\ref{eq:ref_6_2_1}$) по степеням $e^x$ при $x \to -\infty$, то оно качественно
будет подобно ряду (4). При $x \to +\infty$ главный член разложения ($\ref{eq:ref_6_2_1}$) будет
$I_k(x) \sim x^{k+1}$, хотя коэффициент при нем не будет совпадать с точным. Однако
даже такая передача правой асимптоты улучшает точность аппроксимации.
\\

\textbf{Нахождение коэффициентов.} Алгоритмы вычисления коэффициентов рациональных аппроксимаций
обычно достаточно сложны, если добиваться минимизации некоторой нормы
погрешности. Наилучшим был бы алгоритм, обеспечивающий чебышевский альтернанс для относительной погрешности. Однако неясно, как строить такой алгоритм. В данном случае мы ограничились несложным эвристическим алгоритмом, дающим хорошие результаты. Преобразуем ($\ref{eq:ref_6_2_1}$) к следующей
форме:
\begin{equation}
z = \bigg[ \frac{I_k(x)}{\Gamma(k+1)y} \bigg]^{\frac{1}{k}}, z \approx \Bigg( \frac{\sum\limits_{n=0}^{N+1} a_n y^n}{\sum\limits_{n=0}^{N} b_n y^n} \Bigg).
\label{eq:ref_6_2_2}
\end{equation}
Аппроксимация ($\ref{eq:ref_6_2_2}$) содержит $2N + 1$ свободный коэффициент: $a_n$,с $1 \leqslant n \leqslant N + 1$ и $b_m$, с $1 \leqslant m \leqslant N$.
Мы ищем аппроксимацию на интервале $-\infty < x \leqslant 0$, то есть $0 < y \leqslant y_{max} = ln2$. 
Выберем некоторым образом $2N +1$ узлов $y_j : 0 < y_1 < y_2 < ... < y_{2N+1} = y_{max}$ .
По величинам $y_j$ вычислим соответствующие значения $x_j, I_k(x_j)$ и $w_j$. Потребуем, чтобы в $j-$х точках приближенное равенство ($\ref{eq:ref_6_2_2}$) становилось точным, то есть поставим задачу интерполяции
по выбранным узлам.

Заметим, что можно формально взять $j=0$ и положить $y_0 = 0$. Но фактически делать этого
не нужно. В этой точке приближенное равенство ($\ref{eq:ref_6_2_2}$) автоматически становится точным, поскольку аппроксимация ($\ref{eq:ref_6_2_1}$) точно передает главный член левой асимптотики.
\\

\textbf{Система линейных уравнений} для нахождения свободных коэффициентов имеет следующий вид:
\begin{equation}
\sum\limits_{j=1}^{2N+1} AY_j = B,
\label{eq:ref_6_2_3}
\end{equation}
где
\begin{equation}
TODO
\label{eq:ref_6_2_4}
\end{equation}
Обусловленность системы быстро ухудшается с ростом $N$. Тем не менее,
существующая программа дает разумные результаты при её решении. Это может повлиять на величины вычисляемых коэффициентов, однако слабо влияет на погрешность полученных формул (это известный парадокс метода наименьших квадратов). 
%В таблице приведены значения логарифма числа обусловленности ϰ для различных k в зависимости от числа линейно-тригонометрических узлов. Вычисления проводились с помощью встроенной функции cond.

\textbf{Узлы интерполяции.} Положение выбранных узлов сильно влияет на качество полученной интерполяции.
При неудачном положении этих точек могут возникать отрицательные коэффициенты  $a_n, b_m$; это опасно, особенно если знаменатель или числитель обращаются в нуль внутри требуемого диапазона значений $y$.
При наличие таких полученная аппроксимация совершенно неприемлема.
Мы опробовали на практике некоторые способы выбора точек интерполяции. Проиллюстрируем их на примере $k = 2$ и $N = 3$ (это означает 7 свободных коэффициентов); результаты для других $k$ и $N$ были аналогичными.

\textbf{\textit{$1^{\circ}$. Линейное распределение.}} Простейшим способом было линейное
расположение узлов:
\begin{equation}
y_j = \frac{j \cdot y_{max}}{2N+1}, 0 \leqslant j \leqslant 2N +1;
\label{eq:ref_6_2_5}
\end{equation}
напомним, что формально мы можем вводить $y_0$, хотя в расчетах оно не ис-
пользуется. Профиль относительной погрешности $\delta$ для этого случая приве-
ден на рис.№. Видно, что погрешность обращается в нуль во всех узлах интерполяции, а между ними имеет вид полуволн. Амплитуды этих полуволн невелики в средних интервалах, а в крайних они в несколько раз
больше. Расчеты показывают, что при увеличении $N$ отношение погрешности в центре и на периферии быстро возрастает. Это говорит о том, что в середине отрезка следовало бы увеличить расстояние между узлами интерполяции, а на краях отрезка уменьшить.

\textbf{\textit{$2^{\circ}$. Чебышевское распределение.}} Хорошо разработана теория аппроксимаций многочленами, наилучшая в норме $C$. В ней положение узлов интерполяции точно не вычисляется. Однако оно близко к распределению, описываемому тригонометрической функцией:
\begin{equation}
y_j = \frac{1}{2} y_{max} \Bigg(1 - cos\Bigg( \frac{\pi j}{2N+1} \Bigg) \Bigg),0 \leqslant j \leqslant 2N + 1.
\label{eq:ref_6_2_6}
\end{equation}
Профиль погрешности $\delta$ для выбора узлов ($\ref{eq:ref_6_2_6}$) так же показан на рис.2. Между узлами интерполяции он имеет вид полуволн, амплитуды которых велики в середине отрезка и малы по краям. Поэтому для распределения ($\ref{eq:ref_6_2_6}$) надо сблизить узлы интерполяции в середине отрезка и раздвинуть вблизи границ отрезка.

Заметим, что это не противоречит теоретическим результатам для чебышевских приближений: они относятся к аппроксимации многочленами, а мы используем аппроксимацию рациональными функциями.

\textbf{\textit{$3^{\circ}$. Смешанное распределение.}} Представляется естественным построить распределение узлов интерполяции, промежуточное между линейным и тригонометрическим. Такая задача уже возникала в сверхбыстром итерационном методе решения систем эллиптических уравнений на прямоугольной сетке. Воспользуемся предложенным в [ссылка] линейно-тригонометрическим распределением:
\begin{equation}
y_j = \frac{y_{max}}{2} \Bigg[ \frac{2\gamma j}{2N+1} + (1-\gamma)\Bigg(1 - cos \Bigg(\frac{\pi j}{2N + 1}\Bigg) \Bigg)\Bigg], 0 \leqslant j \leqslant 2N+1.
\label{eq:ref_6_2_7}
\end{equation}

Чему равно $\gamma$ ? Пусть есть производящая линейная функция
$F = \gamma F_1 + (1-\gamma)F_2$, где $F_1 = 2\zeta, F_2 = 1 - cos(\pi \zeta), \zeta \epsilon [0;1] \to F_2 \epsilon [0;2]$. Определим производную функции $F$:
\begin{equation}
\frac{dF}{d\zeta} = 2\gamma + \pi (1-\gamma)\sin(\pi \zeta).
\end{equation}
При $\zeta = 0, \frac{dF}{d\zeta} = 2\gamma$, при $\zeta = 0.5, \frac{dF}{d\zeta} = 2\gamma + \pi (1-\gamma)$. Потребуем, чтобы производная в центре была бы вдвое больше, чем производная на границах, то есть $2\frac{dF}{d\zeta}\Bigg|_{\zeta=0} = \frac{dF}{d\zeta}\Bigg|_{\zeta=0.5}$. Это дает $\gamma = \frac{\pi}{2+\pi}$.

Узлы ($\ref{eq:ref_6_2_7}$) были построены для функции, которая фактически является
отношением многочленов одинаковой степени. Поскольку у нас степени
многочленов в числителе и знаменателе отличаются всего на $1$, можно ожидать их хорошей применимости в нашем случае. Результаты расчета также
показаны на рис.№. Видно, что теперь экстремумы в центральных и краевых
интервалах почти одинаковы: отличие составляет $\sim 15\%$. Таким образом получаемое решение близко к чебышевскому альтернансу. Поэтому эвристическое распределение ($\ref{eq:ref_6_2_7}$) можно считать почти неулучшаемым, и пользоваться им для любых $N$ и $k$.

Видно также, что погрешность для линейно-тригонометрического распределения меньше в 7.5 раз, чем для линейного, и в 4 раза по сравнению с тригонометрическим. Это достаточно существенный выигрыш в точности.
\\

\textbf{Влияние числа параметров.} Подробней исследуем погрешность аппроксимации для линейно-
тригонометрических узлов. При $N \leqslant 3$ профили погрешности для всех $k$ и $N$
имеют тот же качественный вид, как и жирная линия на рис.2. Погрешность
обращается в нуль в узлах интерполяции, между ними содержит $2N+1$ полуволн, а экстремумы всех полуволн примерно одинаковы по модулю. Это показывает, что ошибки округления практически не влияют на погрешность аппроксимации.

Картина меняется при $N = 4$. Профиль относительной погрешности становится хаотическим с 
амплитудой $\sim 2 \dot 10^{-16}$ (см. рис.3). Это показывает, что
расчет вышел на ошибки округления, и дальнейшее увеличение числа параметров бессмысленно.

% TODO Рис.3. Профиль погрешности для $k = 2$ и $N = 4$.
В Приложении приведены графики погрешностей для всех значений
$k =1 - 3$ и $N = 1-4$, используемых в данной работе. Они подтверждают сде-
ланные выше выводы.

Погрешность аппроксимации можно характеризовать нормой $C: \delta_C = max\| \delta(y)\|$. 
В табл.2 приведены значения этой величины (точнее, $lg(\delta_C)$
для различных $k$ и $N$). Видно, что максимальная погрешность слабо зависит
от $k$, но быстро убывает с увеличением $N$. Для $N = 1$ аппроксимация дает $\sim 6$
верных десятичных знаков, для $N = 2$ это $\sim 10$ знаков, для $N = 3$ это $\sim 14$ зна-
ков; для $N = 4$ следовало бы ожидать $\sim 18$ знаков, но ошибки округления поз-
воляют выйти всего лишь на $\sim 16$ знаков.
\\


%\S \textbf{6.3. Аппроксимации левых асимпототик полуцелых индексов}