\chapter{АППРОКСИМАЦИИ ФУНКЦИЙ ФД}

\S \textbf{6.1. Глобальные аппроксимации}
\\

Для несложных оценочных расчетов физикам полезно
иметь простые формулы, непрерывно и гладко описывающие функции ФД во всем диапазоне значений аргумента
$-\infty \leqslant x \leqslant +\infty$. Для успешного применения такие формулы должны описывать главные
члены левых и правых асимптотик функций. Физически это обеспечивает переход фермионов в
идеальный газ при высоких температурах, и в полностью вырожденный газ при низких
температурах.

\textbf{Промежуточный аргумент.} Однако, левые асимптотики ($x \to -\infty$) функций ФД являются разложениями по $e^x$. Правые асимптотики ($x \to +\infty$) имеют главный член $tilda x^{k+1}$ и содержат ращложения по степеням $x^{-2}$. Эти разложения сильно различаются по своей математической структуре. Левая асимптотика грубо искажает поведение функции ФД при $x > 0$, а правая при $x < 0$, поэтому построение оценочных формул в виде явной зависимости от $x$ нетривиально. Хорошее решение проблемы удалось найти, введя промежуточный аргумент $y(x)$. В качестве него взята единственная функция ФД, явно выражающаяся через элементарные функции:
\begin{equation}
y(x) \equiv I_0(x) = ln(1+e^x); 0 \leqslant y < +\infty, -\infty < x < +\infty.
\end{equation}
Такой аргумент ведет себя как $e^x$ при $x \to -\infty$, и как $x$ при $x \to +\infty$. Таким образом, он гладко склеивает экспоненту левой асимптотики с полиномом правой асимптотики. Поэтому целесообразно искать аппроксимацию $I_k(x)$ в виде некоторых явных зависимостей от промежуточного аргумента $y$.

Построим большой набор таких приближенных формул.

\textbf{Двучленные формулы.} Двучленная формула была предложена в [8]. Запишем ее в несколько более удобной
форме:
\begin{equation}
I_k(x) \approx \frac{y}{k+1}(\Gamma(k+2)^{1/k} + y)^k, 0 \leqslant y < +\infty.
\label{ref_62}
\end{equation}
Эта формула не содержит подгоночных параметров, её можно применять для любых индексов $k > -2$.

% TODO (перестроить рисунок для всех индексов )
На рис.№. показаны погрешности двучленной формулы ($\ref{ref_62}$) для целых индексов $k$. Видно, что для $k > 0$ эта формула всюду
завышает значение функции. Погрешность её значительна. Максимальное завышение
составляет $30 \% $ для $k = 1$ и доходит до $190 \%$ для $k = 4$. Поэтому формула пригодна
только для очень грубых оценок.

% TODO
\textbf{Трехчленные формулы.} TODO

\textbf{Специальные трехчленные формулы. (Ритус)} TODO

% TODO
\textbf{Пятичленные формулы.} TODO


%\S \textbf{6.2. Аппроксимации левых асимпототик целых индексов}
%\S \textbf{6.3. Аппроксимации левых асимпототик полуцелых индексов}